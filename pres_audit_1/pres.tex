\documentclass{beamer}
\usepackage[utf8]{inputenc}
\usepackage[T1]{fontenc}
\usepackage[francais]{babel} 
\usepackage{lmodern}
\usepackage{hyperref}
\usepackage{tikz}

\usetikzlibrary{trees,shapes.geometric,arrows,decorations.pathmorphing,backgrounds,fit,positioning,shapes.symbols,chains	}
 \tikzset{
    %Define standard arrow tip
    >=stealth',
    %Define style for boxes
    punkt/.style={
           rectangle, dashed,
           rounded corners,
           draw=black, very thin,
           minimum height=2em,
           minimum width = 2cm,
           text centered},
    square/.style={
           rectangle,
           draw=black, thick,
           minimum height=.5cm,
           minimum width = 1cm,
           text centered},
    data/.style={
           rectangle,
           draw=black, thick,
           minimum height= 2cm,
           minimum width = 2cm,
           text centered},
    % Define arrow style
    pil/.style={
           ->,
           thick,
           shorten <=1pt,
           shorten >=1pt,},
    asym/.style={
           <->,
           thin,
           shorten <=1pt,
           shorten >=1pt,
           red!100},
    sym/.style={
           <->,
           thin,
           shorten <=1pt,
           shorten >=1pt,
           blue!100}
}
\usepackage{pgfplots}
\usepackage{eurosym}
\usepackage{rotating}
\usepackage{array}
\usepackage{tikz-uml}
\usepackage{color, colortbl}
\usetheme{Antibes}
\usecolortheme{beaver}
\setbeamertemplate{sections/subsections in toc}[square]
\setbeamertemplate{blocks}[square]%

\author{William Boisseleau -- Pascal Edouard -- Mathieu Latimier -- Julien Legras -- Claire Smets}
\title{Projet annuel - Audit des implantations SSL/TLS}
\titlegraphic{\includegraphics[height=3em]{logo_univ.png}}
\institute{Master 2 Sécurité des Systèmes Informatiques}

\date{20/12/2013}

\begin{document}
{
\setbeamertemplate{headline}[default] 
\begin{frame}
  \titlepage
\end{frame}
}

\frame{
\frametitle{Sommaire}
\tableofcontents
}

\section{Présentation du projet}
\frame{
\frametitle{Présentation du projet}
\begin{block}{Sujet proposé par Ayoub Otmani}
	\begin{enumerate}
		\item audit des clefs cryptographiques
		\item analyse statique de la bibliothèque OpenSSL
		\item analyse dynamique du niveau de sécurité d'un navigateur web
	\end{enumerate}
\end{block}
}

\section{Cas d'utilisation}
\frame[allowframebreaks=1.0]{
\frametitle{Cas d'utilisation}
\begin{tikzpicture}[node distance=-.01cm,font=\tiny,scale=0.8,every node/.style={transform shape}]
\umlactor{Appli RC}
%\umlactor[x=14, y=4]{AC}

\umlusecase[x=3, y=4, width=3cm]{Lire}
\umlusecase[x=3, y=1, width=3cm]{Se connecter}
\umlusecase[x=10, y=2, width=3cm]{Échanger les certificats}
\umlusecase[x=10, y=-1]{Stocker les certificats}

%\umlusecase[x=10, width=3cm]{Envoi message privé}
%\umlusecase[x=10, y=-2, width=3cm]{Envoi message au salon général}
%\umlusecase[x=7, y=-3]{Déconnexion}

\umltrans{Appli RC}{usecase-1}
\umltrans{Appli RC}{usecase-2}
%\umltrans{User}{usecase-3}
\umltrans{Appli RC}{usecase-4}

\umlinclude{usecase-1}{usecase-2}
\umlextend{usecase-3}{usecase-2}

\umlinclude{usecase-4}{usecase-3}
\end{tikzpicture}

\framebreak

\begin{tikzpicture}[node distance=-.01cm,font=\tiny,scale=0.7,every node/.style={transform shape}]
\umlactor{Appli F}
%\umlactor[x=14, y=4]{AC}

\umlusecase[x=4, y=4, width=3cm]{Prendre les moduli}
\umlusecase[x=5, y=2, width=3cm]{Construire l'arbre des produits}
\umlusecase[x=5, y=0, width=3cm]{Construire l'arbre des restes}
\umlusecase[x=6, y=-2]{Sortir la liste des facteurs communs}
\umlusecase[x=12, y=-0.5]{Sortir la liste des moduli vulnérables}

\umltrans{Appli F}{usecase-5}
\umltrans{Appli F}{usecase-6}
\umltrans{Appli F}{usecase-7}
\umltrans{Appli F}{usecase-8}

\umlinclude{usecase-6}{usecase-7}
\umlextend{usecase-5}{usecase-6}
\umlextend{usecase-8}{usecase-7}
\umlinclude{usecase-8}{usecase-9}
%\umlinclude{usecase-4}{usecase-3}
\end{tikzpicture}

\framebreak
\begin{tikzpicture}[node distance=-.01cm,font=\tiny,scale=0.75,every node/.style={transform shape}]

\umlactor{User}
%\umlactor[x=14, y=4]{AC}

\umlusecase[x=4, y=2, width=3cm]{Ouvrir la page web}
\umlusecase[x=4, y=0, width=3cm]{Sélectionner et filtrer les données affichées}
\umlusecase[x=12, y=0, width=3cm]{Exporter les données}
%\umlusecase[x=6, y=-2]{Sortir la liste des facteurs communs}
%\umlusecase[x=12, y=-0.5]{Sortir la liste des moduli vulnérables}

\umltrans{User}{usecase-10}
\umltrans{User}{usecase-11}

\umlextend{usecase-12}{usecase-11}

\end{tikzpicture}

\framebreak

\begin{tikzpicture}[node distance=-.01cm,font=\tiny,scale=0.75,every node/.style={transform shape}]

\umlactor[x=4,y=-1]{Client}
\umlactor[x=15, y=-1]{Serveur}

\umlusecase[x=4, y=2, width=3cm]{Se connecter sur le serveur}
\umlusecase[x=10, y=2, width=3cm]{Établit la connexion}
\umlusecase[x=10, y=-1, width=3cm]{Analyser le niveau de sécurité du navigateur}
\umlusecase[x=10, y=-4]{Afficher les résultats}
%\umlusecase[x=12, y=-0.5]{Sortir la liste des moduli vulnérables}

\umltrans{Client}{usecase-13}
\umltrans{Serveur}{usecase-14}
\umltrans{Serveur}{usecase-15}
\umltrans{Serveur}{usecase-16}

\umlinclude{usecase-13}{usecase-14}
\umlinclude{usecase-14}{usecase-15}
\umlinclude{usecase-15}{usecase-16}

\end{tikzpicture}               
}

\section{Planning de développement}
\frame[allowframebreaks=1.0,font=\tiny]{
\frametitle{Planning de développement}
\begin{block}{Sprint 1 : Récupération des certificats et factorisation}
\begin{itemize}
\item Outils de récupération des certificats SSL et SSH.
\item Outils de factorisation des clefs cryptographiques.
\end{itemize}
\end{block}

\begin{center}
\begin{tikzpicture}

\node[draw=none] (start) {};
\node[square, right=of start] (zmap) {ZMAP};
\node[square,right=of zmap] (applirc) {Appli RC};
\node[square,right=of applirc] (applif) {Appli F};
\node[draw=none] at (10,0) (end) {};

\draw (start) edge[->] node[midway,anchor=south] {ports} (zmap);
\draw (zmap) edge[->] node[midway,anchor=south] {IP} (applirc);
\draw (applirc) edge[->] node[midway,anchor=south] {clefs} (applif);
\draw (applif) edge[->] node[midway,anchor=south] {facteurs} (end);
\draw (applif) edge[] node[midway,anchor=north] {communs} (end);

\end{tikzpicture}
\end{center}

\framebreak

\begin{block}{Sprint 2 : Audit d'OpenSSL}
\begin{itemize}
\item Détermination des fonctions à auditer et des standards associés.
\item Développer des outils de tests d'OpenSSL.
\item Documentation et présentation des résultats de l'analyse.
\end{itemize}
\end{block}

\framebreak
\begin{block}{Sprint 3 : Analyse dynamique des navigateurs web}
\begin{itemize}
\item Serveur HTTPS analysant la poignée de main SSL/TLS.
\item Présentation des résultats de l'analyse.
\end{itemize}
\end{block}
}

\end{document}