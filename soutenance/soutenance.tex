\documentclass{beamer}
\usepackage[utf8]{inputenc}
\usepackage[T1]{fontenc}
\usepackage[francais]{babel} 
\usepackage{lmodern}
\usepackage{hyperref}
\usepackage{tikz}

\usetikzlibrary{trees,shapes.geometric,arrows,decorations.pathmorphing,backgrounds,fit,positioning,shapes.symbols,chains	}
 \tikzset{
    %Define standard arrow tip
    >=stealth',
    %Define style for boxes
    punkt/.style={
           rectangle, dashed,
           rounded corners,
           draw=black, very thin,
           minimum height=2em,
           minimum width = 2cm,
           text centered},
    square/.style={
           rectangle,
           draw=black, thick,
           minimum height=.5cm,
           text centered},
    data/.style={
           rectangle,
           draw=black, thick,
           minimum height= 2cm,
           minimum width = 2cm,
           text centered},
    % Define arrow style
    pil/.style={
           ->,
           thick,
           shorten <=1pt,
           shorten >=1pt,},
    asym/.style={
           <->,
           thin,
           shorten <=1pt,
           shorten >=1pt,
           red!100},
    sym/.style={
           <->,
           thin,
           shorten <=1pt,
           shorten >=1pt,
           blue!100}
}
\usepackage{pgfplots}
\usepackage{eurosym}
\usepackage{rotating}
\usepackage{array}
\usepackage{multicol}

\usetheme{Antibes}
\usecolortheme{beaver}
\setbeamertemplate{sections/subsections in toc}[square]
\setbeamertemplate{blocks}[square]%

\author{Claire Smets -- William Boisseleau -- Pascal Edouard -- Mathieu Latimier -- Julien Legras}
\title{Soutenance projet annuel - Audit des implantations SSL/TLS}
\titlegraphic{\includegraphics[height=3em]{logo_univ.png}}
\institute{Master 2 Sécurité des Systèmes Informatiques}

\date{28/02/2014}

\begin{document}
{
\setbeamertemplate{headline}[default] 
\begin{frame}
  \titlepage
\end{frame}
}

%% 2 MIN -- PASCAL
\section{Introduction}
\subsection{Sujet et problématique}
\frame{
\frametitle{Sujet et problématique}
\begin{block}{Titre du bloc}
	Contenu
\end{block}
}

\frame{
\frametitle{Sommaire}
\begin{multicols}{2}
\tableofcontents
\end{multicols}
}

%% PARTIE 1 - 16 MIN
\section{Audit des clefs RSA des certificats}

%% 4 MIN -- CLAIRE
\subsection{Récupération}
\subsubsection{Adresses}
\frame{
    \frametitle{Récupération des adresses 1}
    	\begin{block}{ZMAP}
	\begin{itemize}
		\item open source ;
		\item outil de scan réseau ;
		\item adresses IPv4 ;
		\item paquets SYN sur le port 443.\\
	\end{itemize}
	\end{block}
}

\subsubsection{Certificats}
\frame	{
    \frametitle{Récupération des certificats 1}
    %% Certificats + clefs
    \begin{block}{Application Récupération de Certificats}
	\begin{itemize}
		\item script perl ;
		\item certificats SSL ;
		\item stocker l'ensemble des empreintes dans un dossier : 
		\begin{itemize}
			\item certificats ;
			\item clefs de session.\\
		\end{itemize}	
	\end{itemize}
	\end{block}
}

\frame{
    \frametitle{Récupération des certificats 2}
    	\begin{block}{Algorithme}
		algo de récupération des certificats	
	\end{block}
}

\frame{
    \frametitle{Gestion des doublons 1}
	\begin{block}{Gestion des doublons}
	\begin{itemize}
		\item script perl ;
		\item si une empreinte de trouve déjà dans le dossier : on la stocke dans un autre dossier, celui des doublons ;
		\item liens symboliques.\\
	\end{itemize}
	\end{block}	
}

\frame{
    \frametitle{Gestion des doublons 2}
	\begin{block}{Algorithme gestion des doublons}
		algo de gestion des doublons
	\end{block}	
}

%% 5 MIN
\subsection{Factorisation}
\frame{
    \frametitle{Factorisation}
    %% ALGOS -- WILLIAM
    %% OPTIM -- JULIEN
}
\frame{
    \frametitle{Factorisation -- Démonstration}

}

%% 5 MIN -- PASCAL
\subsection{Résultats}
\frame{
    \frametitle{Résultats}
    %% site : stats
}


%% PARTIE 2 - 16 MIN
\section{Audit d'OpenSSL}
\frame{
%% 1 MIN -- CLAIRE
\frametitle{Introduction}
	\begin{block}{Contexte}
	\begin{itemize}
		\item récent scandal sur la NSA ;
		\item beaucoup d'outils utilisés. \\
	\end{itemize}
	Mais à qui peut-on faire confiance?
	\end{block}
	\begin{block}{Audit d'OpenSSL}
		Cinq grands axes : 
		\begin{itemize}
			\item l'entropie ;
			\item la génération des clefs ;
			\item le chiffrement et les protocoles ;
			\item les signatures et les authentifications ;
			\item les protocoles SSL et TLS.\\
		\end{description}
	\end{block}
}
%% 4 MIN -- WILLIAM (3 MIN)
\subsection{Entropie}
\frame{
\frametitle{Entropie}

}

%% MATHIEU (1 MIN)
\frame{
\frametitle{Entropie -- Démonstration}

}

%% 2 MIN -- MATHIEU
\subsection{Génération des clefs}
\frame{
\frametitle{Génération des clefs}

}

%% 2 MIN -- CLAIRE
\subsection{Chiffrement et protocoles}
\frame{
\frametitle{Chiffrement et protocoles}

}

%% 2 MIN -- MATHIEU
\subsection{Signature et authentification}
\frame{
\frametitle{Signature et authentification}

}

%% 3 MIN -- JULIEN
\subsection{Protocole SSL/TLS}
\frame{
\frametitle{Protocole SSL/TLS}

}

%% 2 MIN -- CLAIRE
\subsection{Ouverture}
\frame{
\frametitle{Ouverture}

}


%% PARTIE 3  - 8 MIN
\section{Analyse dynamique du navigateur client}
%% 2 MIN 30 -- WILLIAM
\subsection{Faiblesses identifiées}
\frame{
\frametitle{Faiblesses identifiées}

}

%% 2 MIN 30 --  JULIEN
\subsection{Implémentation}
\frame{
\frametitle{Implémentation}

}

%% 3 MIN -- MATHIEU
\subsection{Démonstration}
\frame{
\frametitle{Démonstration}
%% Chrome
%% Lynx 
%% s_client
}

%% 2 MIN -- PASCAL
\section{Conclusion}
\frame{
\frametitle{Conclusion}
}
\end{document}
