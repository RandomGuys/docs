\documentclass{beamer}
\usepackage[utf8]{inputenc}
\usepackage[T1]{fontenc}
\usepackage[francais]{babel} 
\usepackage{lmodern}
\usepackage{hyperref}
\usepackage{tikz}

\usetikzlibrary{trees,shapes.geometric,arrows,decorations.pathmorphing,backgrounds,fit,positioning,shapes.symbols,chains	}
 \tikzset{
    %Define standard arrow tip
    >=stealth',
    %Define style for boxes
    punkt/.style={
           rectangle, dashed,
           rounded corners,
           draw=black, very thin,
           minimum height=2em,
           minimum width = 2cm,
           text centered},
    square/.style={
           rectangle,
           draw=black, thick,
           minimum height=.5cm,
           text centered},
    data/.style={
           rectangle,
           draw=black, thick,
           minimum height= 2cm,
           minimum width = 2cm,
           text centered},
    % Define arrow style
    pil/.style={
           ->,
           thick,
           shorten <=1pt,
           shorten >=1pt,},
    asym/.style={
           <->,
           thin,
           shorten <=1pt,
           shorten >=1pt,
           red!100},
    sym/.style={
           <->,
           thin,
           shorten <=1pt,
           shorten >=1pt,
           blue!100}
}
\usepackage{pgfplots}
\usepackage{eurosym}
\usepackage{rotating}
\usepackage{array}
\usepackage{multicol}

\usetheme{Antibes}
\usecolortheme{beaver}
\setbeamertemplate{sections/subsections in toc}[square]
\setbeamertemplate{blocks}[square]%

\author{Claire Smets -- William Boisseleau -- Pascal Edouard -- Mathieu Latimier -- Julien Legras}
\title{Soutenance projet annuel - Audit des implantations SSL/TLS}
\titlegraphic{\includegraphics[height=3em]{logo_univ.png}}
\institute{Master 2 Sécurité des Systèmes Informatiques}

\date{28/02/2014}

\begin{document}
{
\setbeamertemplate{headline}[default] 
\begin{frame}
  \titlepage
\end{frame}
}

%% 2 MIN -- PASCAL
\section{Introduction}
\subsection{Sujet et problématique}
\frame{
\frametitle{Sujet et problématique}
\begin{block}{Titre du bloc}
	Contenu
\end{block}
}

\frame{
\frametitle{Sommaire}
\begin{multicols}{2}
\tableofcontents
\end{multicols}
}

%% PARTIE 1 - 16 MIN
\section{Audit des clefs RSA des certificats}

%% 4 MIN -- CLAIRE
\subsection{Récupération}
\subsubsection{Adresses}
\frame{
    \frametitle{Récupération des adresses 1}
    	\begin{block}{ZMAP}
	\begin{itemize}
		\item open source ;
		\item outil de scan réseau ;
		\item adresses IPv4 ;
		\item paquets SYN sur le port 443.\\
	\end{itemize}
	\end{block}
}

\subsubsection{Certificats}
\frame	{
    \frametitle{Récupération des certificats 1}
    %% Certificats + clefs
    \begin{block}{Application Récupération de Certificats}
	\begin{itemize}
		\item script perl ;
		\item certificats SSL ;
		\item stocker l'ensemble des empreintes dans un dossier : 
		\begin{itemize}
			\item certificats ;
			\item clefs de session.\\
		\end{itemize}	
	\end{itemize}
	\end{block}
}

\frame{
    \frametitle{Récupération des certificats 2}
    	\begin{block}{Algorithme}
		algo de récupération des certificats	
	\end{block}
}

\frame{
    \frametitle{Gestion des doublons 1}
	\begin{block}{Gestion des doublons}
	\begin{itemize}
		\item script perl ;
		\item si une empreinte de trouve déjà dans le dossier : on la stocke dans un autre dossier, celui des doublons ;
		\item liens symboliques.\\
	\end{itemize}
	\end{block}	
}

\frame{
    \frametitle{Gestion des doublons 2}
	\begin{block}{Algorithme gestion des doublons}
		algo de gestion des doublons
	\end{block}	
}

%% 5 MIN
\subsection{Factorisation}
\frame{
    \frametitle{Factorisation}
    %% ALGOS -- WILLIAM
    %% OPTIM -- JULIEN
}
\frame{
    \frametitle{Factorisation -- Démonstration}

}

%% 5 MIN -- PASCAL
\subsection{Résultats}
\frame{
    \frametitle{Résultats}
    %% site : stats
}


%% PARTIE 2 - 16 MIN
\section{Audit d'OpenSSL}
\frame{
%% 1 MIN -- CLAIRE
\frametitle{Introduction}
	\begin{block}{Contexte}
	\begin{itemize}
		\item récent scandal sur la NSA ;
		\item beaucoup d'outils utilisés. \\
	\end{itemize}
	Mais à qui peut-on faire confiance?
	\end{block}
	\begin{block}{Audit d'OpenSSL}
		Cinq grands axes : 
		\begin{itemize}
			\item l'entropie ;
			\item la génération des clefs ;
			\item le chiffrement et les protocoles ;
			\item les signatures et les authentifications ;
			\item les protocoles SSL et TLS.\\
		\end{itemize}
	\end{block}
}
%% 4 MIN -- WILLIAM (3 MIN)
\subsection{Entropie}
\frame{
\frametitle{Entropie}

}

%% MATHIEU (1 MIN)
\frame{
\frametitle{Entropie -- Démonstration}
\begin{block}{Faille Debian 4.0 sous OpenSSL 0.9.8}
Après avoir récupéré l'ensemble des certificats sur l'Internet, on peut identifier rapidement ceux qui ont étés générés durant la faille Debian/OpenSSL entre 2006 et 2008.\\
\begin{itemize}
\item \textbf{Durée de l'attaque} : quelques heures
\item \textbf{Conséquences} : forger de faux certificats, de fausses signatures et déchiffrer des messages privées.
\item \textbf{Qui?} : grandes entreprises (e.g. IBM, CISCO), routeurs, universités, etc.
\item \textbf{Fin de validité de certificats} : Entre 2020 et 2030.
\end{itemize}
\end{block}
}

%% 2 MIN -- MATHIEU
\subsection{Génération des clefs}
\frame{
\frametitle{Génération des clefs}
\begin{block}{Principes de Kerchkoff}
	\begin{itemize}
	\item Le \textit{secret} réside dans la clef ;
	\item Les algorithmes de génération de clés ne doivent donner aucune information sur la clé.
	\end{itemize}
\end{block}

\begin{block}{Deux grands types de générateur de clés}
\begin{itemize}
\item Générateurs de bits aléatoires (RGB) pour les clés privées de certains algorithmes (i.e. AES, DSA) ou pour le salage (i.e. \textit{seed} de RSA)
\item Générateurs de clefs asymétriques, qui utilisent des fonctions à sens uniques, largement diffusées et ne devant délivrer aucune information sur le secret.
\end{itemize}
\end{block}
}

\frame{
\frametitle{Génération des clefs}
\begin{block}{Audit : Diffie-Hellman Ephémère en mode FIPS}
\begin{itemize}
\item \textbf{Description} : Un attaquant écoutant une communication chiffré en SSL/TLS entre un client et un serveur peut déchiffrer tout les messages en forçant la génération d'un secret Diffie-Hellman prédictible.
\item \textbf{Comment?} : En modifiant le trafic réseau par exemple.
\item \textbf{Pourquoi?} : L'activation du mode FIPS ne rejette pas les paramètres P/Q faibles pour les algorithmes EDH/DHE.
\item \textbf{Où?} : Dans \texttt{crypto/dh/dh\_key.c} une partie de code génère un faux positif dans certains cas (sur une condition de test).
\item \textbf{Solution :} Logiciel \textit{Nessus Vulnerability Scanner} pour tester la configuration des serveurs.
\end{itemize}
\end{block}
}

%% 2 MIN -- CLAIRE
\subsection{Chiffrement et protocoles}
\frame{
\frametitle{Chiffrement et protocoles}

}

%% 2 MIN -- MATHIEU
\subsection{Signature et authentification}
\frame{
\frametitle{Signature et authentification}
\begin{block}{Définition}
\begin{itemize}
\item Juridiquement, une signature électronique à même valeur qu'une signature manuscrite.
\item Elle \textbf{DOIT} assurer l'intégrité, l'authentification et la non-répudiation d'un message.
\end{itemize}
\end{block}

Des anciennes versions d'OpenSSL ont des vulnérabilités au niveau de la vérification de messages signés, ou des fuites d'informations sur la clé privée ayant servi à chiffrer.
}

\frame{
\frametitle{Signature et authentification}
\begin{block}{Audit : Attaque par injection de fautes sur les certificats RSA.}
\begin{itemize}
\item \textbf{Description} : L'attaque se fait sur des morceaux de la signature récupéré afin de récupérer la clé privée bit à bit.
\item \textbf{Comment?} : Du bon matériel, surtout au niveau de la mémoire vive (i.e. Système Linux avec une architecture SPARC) et un oracle (e.g. système de prédictions) permettant de fabriquer la clé.
\item \textbf{Temps de l'attaque} : une centaine d'heure.
\end{itemize}
\end{block}
}

\frame{
\frametitle{Signature et authentification}
\begin{block}{Audit : Attaque par injection de fautes sur les certificats RSA.} 
\begin{itemize}
\item \textbf{Un problème dans le code OpenSSL?} : La fonction \texttt{Fixed\_Window\_Exponentiation} utilise des milliers de multiplications, qui est opération la plus sensible en cas de dégradation du micro-processeur.
\item \textbf{Solution :} Aucune! On pourrait utiliser la technique du \texttt{square\_and\_multiply} mais elle a l'inconvénient d'être vulnérable à une attaque par \textit{timing}.
\item \textbf{Conséquences :} A moins que l'attaquant n'ai accès physiquement à votre machine les risques sont faibles. Cependant l'Université du Michigan cherche un moyen de faire des injections à distance à base d'\texttt{impulsions lasers}.
\end{itemize}
\end{block}
}

%% 3 MIN -- JULIEN
\subsection{Protocole SSL/TLS}
\frame{
\frametitle{Protocole SSL/TLS}

}

%% 2 MIN -- CLAIRE
\subsection{Ouverture}
\frame{
\frametitle{Ouverture}

}


%% PARTIE 3  - 8 MIN
\section{Analyse dynamique du navigateur client}
%% 2 MIN 30 -- WILLIAM
\subsection{Faiblesses identifiées}
\frame{
\frametitle{Faiblesses identifiées}

}

%% 2 MIN 30 --  JULIEN
\subsection{Implémentation}
\frame{
\frametitle{Implémentation}

}

%% 3 MIN -- MATHIEU
\subsection{Démonstration}
\frame{
\frametitle{Démonstration}
\begin{block}{Analyse de la sécurité des navigateurs clients}
\begin{itemize}
%% Chrome
\item Sous le navigateur graphique \texttt{Chrome} : le plus utilisé dans le monde.
%% Lynx  
\item Sous le navigateur console \texttt{lynx} : apprécié chez les développeurs.\newline
\end{itemize}
\end{block}

\begin{block}{Modification manuelle}
Nous pouvons modifier la \textit{ciphersuite} du client avec la commande \texttt{s\_client}.
\end{block}
}

%% 2 MIN -- PASCAL
\section{Conclusion}
\frame{
\frametitle{Conclusion}
}
\end{document}
