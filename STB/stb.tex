\documentclass[a4paper,11pt,french]{article}
\usepackage[utf8]{inputenc}

\usepackage[T1]{fontenc}
\usepackage[francais]{babel} 
\usepackage[top=1.5cm, bottom=2cm, left=1.5cm, right=1.5cm, includeheadfoot]{geometry} %pour les marges
\usepackage{lmodern}
\usepackage{pict2e}
\usepackage{tikz}	
\usepackage{tikz-uml}
\usepackage{fancyhdr} % Required for custom headers
\usepackage{lastpage} % Required to determine the last page for the footer
\usepackage{extramarks} % Required for headers and footers
\usepackage{graphicx} % Required to insert images
\usepackage{tabularx, longtable}
\usepackage{color, colortbl}

\usepackage{cmbright}
\usepgflibrary{arrows} % for pgf-umlsd

\linespread{1.1} % Line spacing

% Set up the header and footer
\pagestyle{fancy}
\lhead{\textbf{\hmwkClass -- \hmwkSubject \\ \hmwkTitle \\ \hmwkDocName}} % Top left header
\rhead{\includegraphics[width=10em]{logo_univ.png}}
\lfoot{\lastxmark} % Bottom left footer
\cfoot{} % Bottom center footer
\rfoot{Page\ \thepage\ / \pageref{LastPage}} % Bottom right footer
\renewcommand\headrulewidth{0.4pt} % Size of the header rule
\renewcommand\footrulewidth{0.4pt} % Size of the footer rule

\setlength{\headheight}{40pt}

\newcommand{\hmwkTitle}{Audit des implantations SSL/TLS} % Assignment title
\newcommand{\hmwkClass}{Master 2 SSI } % Course/class
\newcommand{\hmwkAuthorName}{Claire \sc{SMETS}} % Your name
\newcommand{\hmwkSubject}{Conduite de projet} % Subject
\newcommand{\hmwkDocName}{Spécification technique des besoins} % Document name

\newcommand{\version}{1.0} % Document version
\newcommand{\docDate}{\today} % Document date
\newcommand{\checked}{Edouard PASCAL} % Checker name
\newcommand{\approved}{Mr OTMANI} % Approver name

\definecolor{gris}{rgb}{0.95, 0.95, 0.95}

\author{\hmwkAuthorName}
\date{} % Insert date here if you want it to appear below your name

\newcommand{\fiche}[9] {
	\noindent
\begin{tabular}{|p{4cm}| p{2cm} | p{4cm} | p{.5cm} | p{7cm}|} 
\hline
\rowcolor{blue}
\multicolumn{2}{|l|}{\color{white}\bfseries{Nom}} & \multicolumn{3}{l|}{\color{white}\bfseries{#1}}\\
\hline
\multicolumn{2}{|l|}{\bfseries{Acteurs concernés}} & \multicolumn{3}{m{10.5cm}|}{#2}\\
\hline
\multicolumn{2}{|l|}{\bfseries{Description}} & \multicolumn{3}{m{10.5cm}|}{#3}\\
\hline
\multicolumn{2}{|l|}{\bfseries{Préconditions}} & \multicolumn{3}{m{10.5cm}|}{#4}\\
\hline
\multicolumn{2}{|l|}{\bfseries{Evénements déclenchants}} & \multicolumn{3}{m{10.5cm}|}{#5}\\
\hline
\multicolumn{2}{|l|}{\bfseries{Conditions d'arrêt}} & \multicolumn{3}{m{10.5cm}|}{#6}\\
\hline
\rowcolor{gray}
\multicolumn{5}{|c|}{\bfseries{Description du flot d'événements principal}}\\
\hline
\rowcolor{gray}
\multicolumn{3}{|c|}{\bfseries{Acteur(s)}} & \multicolumn{2}{c|}{\bfseries{Système}}\\
\hline
\multicolumn{3}{|p{7.5cm}|}{#7} & \multicolumn{2}{p{7.5cm}|}{#8}\\
\hline
\multicolumn{2}{|l}{\bfseries{Flots d'exceptions}} & \multicolumn{3}{|p{11.5cm}|}{#9}\\
\hline
\end{tabular}
\\
}

\definecolor{gris}{rgb}{0.95, 0.95, 0.95}

\begin{document}
\pagestyle{fancy}

\vspace*{5cm}
\begin{center}\textbf{\Huge{\hmwkDocName}}\end{center}
\vspace*{7cm}
	
\begin{center}
\fcolorbox{black}{gris}{
\begin{minipage}{10cm}
\begin{tabularx}{10cm}{lXl}
	\bfseries{Version} & & \version\\
	& & \\
	\bfseries{Date} & & \docDate\\
	& & \\
	\bfseries{Rédigé par} & & \hmwkAuthorName \\
	& & \\
	\bfseries{Relu par} & & \checked \\
	& & \\
	\bfseries{Approuvé par} & & \approved \\
	& & \\
\end{tabularx}
\end{minipage}
}
\end{center}

\newpage

%Tableau de mises à jour
\vspace*{1cm}
\begin{center}
\textbf{\huge{MISES À JOUR}}\\
\vspace*{3cm}
	\begin{tabularx}{16cm}{|c|c|X|}
	\hline
	\bfseries{Version} & \bfseries{Date} & \bfseries{Modifications réalisées}\\
	\hline
	0.1 & 02/12/2013 & Création du Document\\
	\hline
	1.0 & 12/12/2013 & Présentation de la première version\\
	\hline
	\end{tabularx}
\end{center}

%La table des matières
\clearpage
\tableofcontents
\clearpage

% OBJET
\section{Objet}
Le projet s'inscrit dans un cadre relatif à un fait d'actualité récemment mis à jour, lié aux révélations de Snowden sur les pratiques de la NSA. Le monde de la cryptographie est en proie à de grandes incertitudes suite à ces révélations et remet en question tous les systèmes jusqu'alors développés. Notons que ce ne sont pas les algorithmes des systèmes cryptographiques qui sont remis en cause, mais leur développement machine qui n'est pas (ou plus) considéré comme nécessairement sûr. La NSA a par exemple très bien pu introduire des backdoors, ou faiblesses dans les logiciels, de telle sorte qu'ils puissent accéder aux données claires sans diffcultés. On peut par exemple remettre en questions les outils développ és par des laboratoires tels que RSA labs, ou même encore des standards proposés par des agences comme le NIST (Openssl, AES, sont-ils vraiment sûrs?)

\textbf{Objectif technique : }\\
\begin{itemize}
	\item coder un outil permettant de récupérer un grand nombre de certificats
	\item coder un outil permettant de trouver d'éventuels facteurs communs aux certificats récupérés
	\item  d'implémenter un système serveur permettant l'audit des clients. (éventuellement, selon le temps restant)
\end{itemize}

\textbf{Résultat attendu : }\\
Une étude sur les certificats utilisés sur internet, ainsi que l'imlantation d'OpenSSL très utilisé.

% DOCUMENTS APPLICABLES ET DE RÉFÉRENCE
\section{Documents applicables et de référence}
\begin{itemize}
	\item Appel d'offre : Audit des implantations SSL/TLS
	\item les standards cryptographiques
\end{itemize}

% TERMINOLOGIE
\section{Terminologie et sigles utilisés}
\begin{itemize}
	\item \textbf{Appli F} : Application de Factorisation
	\item \textbf{Appli RC} : Application de Récupération des Certificats
	\item \textbf{Audit} : Analyse effectuée par un passage en revue complet et minutieux. Dans le cadre de notre projet, ce sera le code d’OpenSSL.
	\item \textbf{Certificat} : Un certificat électronique (aussi appelé certificat numérique ou certificat de clé publique) peut être vu comme une carte d'identité numérique. Il est utilisé principalement pour identifier une entité physique ou morale, mais aussi pour chiffrer des échanges. \\Il est signé par un tiers de confiance qui atteste du lien entre l'identité physique et l'entité numérique (Virtuel).
	\item \textbf{Clefs} : Nous parlons ici des clefs RSA.
	\item \textbf{Machine D} : Machine Distante
	\item \textbf{Machine L} : Machine Locale
	\item \textbf{Modulus} : En mathématiques et plus précisément en théorie algébrique des nombres, l’arithmétique modulaire est un ensemble de méthodes permettant la résolution de problèmes sur les nombres entiers. Ces méthodes dérivent de l’étude du reste obtenu par une division euclidienne.
\end{itemize}

% EXIGENCES FONCTIONNELLES
\section{Exigences fonctionnelles}
\subsection{Présentation de la mission du produit logiciel}
%\begin{itemize}
%\item [EF.x] : xxx ;
%\end{itemize}

\begin{center}
\begin{tabular}{|l|p{6cm}|p{6cm}|l|}
\hline
\bfseries{Id} & \bfseries{Intitulé} & \bfseries{Acteur(s)} & \bfseries{Priorité}\\
\hline
UC.1 & Récupération des certificats & Appli RC, Machine D & Indispensable\\
\hline
UC.2 & Factoridation des moduli des certificats & Appli F & Indispensable\\
\hline
UC.3 & Présentation des résultats & Utilisateur & Indispensable\\
\hline
UC.4 & Audit d'OpenSSL & Utilisateur & Indispensable\\
\hline
UC.5 & Évaluation du niveau de sécurité du navigateur client & Client et Serveur & Optionnel\\
\hline
\end{tabular}
\end{center}


%\begin{itemize}
%\item [RG.X] xxx
%\end{itemize}
\subsection{UC 1 : Récupération des certificats}

\fiche
	{Récupération des certificats} %nom
	{Appli RC, Machine D} %acteurs concernés
	{Récupérer une liste des certificats sur les adresses et les ports ouverts, scannés par ZMAP} %description
	{Avoir une liste de machines distantes dont les ports sont ouverts} %préconditions
	{L'administrateur lance l'application} % événements déclenchants
	{La liste est parcourue en entier} %conditions d'arret
	{\begin{itemize}  %flot d'événements (acteurs)
		\item [1.] Lire la liste
		\item [2.] Se connecter à la machine distante
		\item [3.] Récupérer les certificats
		\item [4.] Stocker le certificat
	 \end{itemize}
	} 
	{\begin{itemize}  %flot d'événements (systeme)
		\item []
		\item []
		\item [3.] Récupérer les certificats
	\end{itemize}
	 }
	{Coupure de connexion : retenter une deuxième fois si la Machine distante ne répond pas} %Flots d'exceptions (RG)


\subsection{UC 2 : Factorisation des moduli des certificats}

\fiche
	{Factorisation des moduli des certificats} %nom
	{Appli F} %acteurs concernés
	{Pour chaque certificat obtenu, on essaie de lui trouver des facteurs communs dans son modulus} %description
	{Avoir une liste de certificats et donc leurs moduli} %préconditions
	{L'application est lancée} % événements déclenchants
	{La liste est parcourue en entier} %conditions d'arret
	{\begin{itemize}  %flot d'événements (acteurs)
		\item [1.] Prendre les moduli
		\item [2.] Construire l'arbre des produits
		\item [3.] Construire l'arbre des restes
		\item [4.] Sortir la liste des facteurs communs
		\item [5.] Sortir la liste des moduli vulnérables trouvés
	 \end{itemize}
	} 
	{\begin{itemize}  %flot d'événements (systeme)
		\item []
		\item []
	\end{itemize}
	 }
	{Pas assez de ressources : Stocker l'arbre en dur au fur et à mesure.} %Flots d'exceptions (RG)


\subsection{UC 3 : Présentation des résultats}

\fiche
	{} %nom
	{Utilisateur} %acteurs concernés
	{Affiche les résultats produits par les précédentes fonctions} %description
	{Des résultats ont été produits} %préconditions
	{Demande d'affchage des ces résultats} % événements déclenchants
	{Demande d'arrêt d'affichage des résultats} %conditions d'arret
	{\begin{itemize}  %flot d'événements (acteurs)
		\item [1.] Ouvrir la page web
		\item [2.] Sélectionner et filtrer les données affchées
		\item [3.] Exporter les données
	 \end{itemize}
	} 
	{\begin{itemize}  %flot d'événements (systeme)
		\item []
		\item []
	\end{itemize}
	 }
	{} %Flots d'exceptions (RG)

\subsection{UC 4 : Audit d'OpenSSL}
\begin{enumerate}
 	\item Nous étudierons tout d'abord les recommandations et les standards actuels (cf. PKCS) sous forme d'état de l'art, et nous en choisirons de façon argumentée un standard sur lequel nous baserons nos analyses.
	\item Nous analyserons dans un deuxième temps le code source d'OpenSSL (la dernière version), en particulier sa partie développant la génération d'éléments aléatoires et ses primitives cryptographiques. Nous en ferons de plus une description détaillée. Nous comparerons ensuite nos résultats avec les recommandations choisies, puis nous conclurons en déterminant si cette génération est efficace.
\end{enumerate}


\subsection{UC 5 : Évaluation du niveau de sécurité du navigateur client}
Suivant le temps restant, il nous est proposé d'implémenter un système serveur permettant l'audit des clients. Le logiciel sur le serveur doit évaluer la sécurité du client suivant un protocole établi, en analysant son système et/ou lui prop osant des challenges spécifiques, en particulier autour de SSL. Le logiciel devra rendre un diagnostic et un rapport sur les différents éléments analysés, afin de permettre au client de constater des problèmes lorsqu'il y en a, et d'en connaître leur provenance.

\section{Diagrammes de cas d'utilisation}

\subsection{UC 1}

\setlength{\fboxrule}{0.1mm}

\begin{tikzpicture}
\umlactor{Appli RC}
%\umlactor[x=14, y=4]{AC}

\umlusecase[x=3, y=4, width=3cm]{Lire}
\umlusecase[x=3, y=1, width=3cm]{Se connecter}
\umlusecase[x=10, y=2, width=3cm]{Échanger les certificats}
\umlusecase[x=10, y=-1]{Stocker les certificats}

%\umlusecase[x=10, width=3cm]{Envoi message privé}
%\umlusecase[x=10, y=-2, width=3cm]{Envoi message au salon général}
%\umlusecase[x=7, y=-3]{Déconnexion}

\umltrans{Appli RC}{usecase-1}
\umltrans{Appli RC}{usecase-2}
%\umltrans{User}{usecase-3}
\umltrans{Appli RC}{usecase-4}

\umlinclude{usecase-1}{usecase-2}
\umlextend{usecase-3}{usecase-2}

\umlinclude{usecase-4}{usecase-3}
\end{tikzpicture}

\subsection{UC 2}

\begin{tikzpicture}
\umlactor{Appli F}
%\umlactor[x=14, y=4]{AC}

\umlusecase[x=4, y=4, width=3cm]{Prendre les moduli}
\umlusecase[x=5, y=2, width=3cm]{Construire l'arbre des produits}
\umlusecase[x=5, y=0, width=3cm]{Construire l'arbre des restes}
\umlusecase[x=6, y=-2]{Sortir la liste des facteurs communs}
\umlusecase[x=12, y=-0.5]{Sortir la liste des moduli vulnérables}

\umltrans{Appli F}{usecase-5}
\umltrans{Appli F}{usecase-6}
\umltrans{Appli F}{usecase-7}
\umltrans{Appli F}{usecase-8}

\umlinclude{usecase-6}{usecase-7}
\umlextend{usecase-5}{usecase-6}
\umlextend{usecase-8}{usecase-7}
\umlinclude{usecase-8}{usecase-9}
%\umlinclude{usecase-4}{usecase-3}
\end{tikzpicture}

\subsection{UC 3}
\begin{tikzpicture}

\umlactor{User}
%\umlactor[x=14, y=4]{AC}

\umlusecase[x=4, y=2, width=3cm]{Ouvrir la page web}
\umlusecase[x=4, y=0, width=3cm]{Sélectionner et filtrer les données affichées}
\umlusecase[x=12, y=0, width=3cm]{Exporter les données}
%\umlusecase[x=6, y=-2]{Sortir la liste des facteurs communs}
%\umlusecase[x=12, y=-0.5]{Sortir la liste des moduli vulnérables}

\umltrans{User}{usecase-10}
\umltrans{User}{usecase-11}

\umlextend{usecase-12}{usecase-11}

\end{tikzpicture}

\subsection{UC 5}
\begin{tikzpicture}

\umlactor{Client}
\umlactor[x=15, y=-1]{Serveur}

\umlusecase[x=4, y=2, width=3cm]{Se connecter sur le serveur}
\umlusecase[x=10, y=2, width=3cm]{Établit la connexion}
\umlusecase[x=10, y=-1, width=3cm]{Analyser le niveau de sécurité du navigateur}
\umlusecase[x=10, y=-4]{Afficher les résultats}
%\umlusecase[x=12, y=-0.5]{Sortir la liste des moduli vulnérables}

\umltrans{Client}{usecase-13}
\umltrans{Serveur}{usecase-14}
\umltrans{Serveur}{usecase-15}
\umltrans{Serveur}{usecase-16}

\umlinclude{usecase-13}{usecase-14}
\umlinclude{usecase-14}{usecase-15}
\umlinclude{usecase-15}{usecase-16}

\end{tikzpicture}


\section{Exigences opérationnelles}
\begin{enumerate}
	\item [EO.X] : Lors de la récupération des certificats, les protocoles visés seront SSH et SSL/TLS (cf UC1)
\end{enumerate}


\section{Exigences d'interface}
\begin{enumerate}
	\item [EI.1] : Présentation des résultats de manière claire et lisible.
\end{enumerate}

\section{Exigences de qualité}
\begin{enumerate}
	\item [EQ.1] : Application optimisée de la factorisation des clefs.
	\item [EQ.2] : Présentation ergonomique des résultats obtenus.
\end{enumerate}

\section{Exigences de réalisation}
\begin{enumerate}
	\item [ER.1] : Le programme de factorisation des clefs de certificat.
	\item [ER.2] : Analyse des résultats obtenus sur la factorisation des clefs.
	\item [ER.3] : Document détaillant l'audit réalisé sur OpenSSL.
\end{enumerate}

\end{document}






