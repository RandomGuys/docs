\documentclass[a4paper,11pt,french]{article}
\usepackage[utf8]{inputenc}

\usepackage[T1]{fontenc}
\usepackage[francais]{babel} 
\usepackage[top=1.5cm, bottom=2cm, left=1.5cm, right=1.5cm, includeheadfoot]{geometry} %pour les marges
\usepackage{lmodern}
\usepackage{pict2e}
\usepackage{tikz}	
\usepackage{tikz-uml}
\usepackage{fancyhdr} % Required for custom headers
\usepackage{lastpage} % Required to determine the last page for the footer
\usepackage{extramarks} % Required for headers and footers
\usepackage{graphicx} % Required to insert images
\usepackage{tabularx, longtable}
\usepackage{color, colortbl}

\usepackage{cmbright}
\usepgflibrary{arrows} % for pgf-umlsd

\linespread{1.1} % Line spacing

% Set up the header and footer
\pagestyle{fancy}
\lhead{\textbf{\hmwkClass -- \hmwkSubject \\ \hmwkTitle \\ \hmwkDocName}} % Top left header
\rhead{\includegraphics[width=10em]{logo_univ.png}}
\lfoot{\lastxmark} % Bottom left footer
\cfoot{} % Bottom center footer
\rfoot{Page\ \thepage\ / \pageref{LastPage}} % Bottom right footer
\renewcommand\headrulewidth{0.4pt} % Size of the header rule
\renewcommand\footrulewidth{0.4pt} % Size of the footer rule

\setlength{\headheight}{40pt}

\newcommand{\hmwkTitle}{Audit des implantations SSL/TLS} % Assignment title
\newcommand{\hmwkClass}{Master 2 SSI } % Course/class
\newcommand{\hmwkAuthorName}{X} % Your name
\newcommand{\hmwkSubject}{Conduite de projet} % Subject
\newcommand{\hmwkDocName}{Spécification technique des besoins} % Document name

\newcommand{\version}{X} % Document version
\newcommand{\docDate}{X} % Document date
\newcommand{\checked}{X} % Checker name
\newcommand{\approved}{X} % Approver name

\definecolor{gris}{rgb}{0.95, 0.95, 0.95}

\author{\hmwkAuthorName}
\date{} % Insert date here if you want it to appear below your name

\newcommand{\fiche}[9] {
	\noindent
\begin{tabular}{|p{4cm}| p{2cm} | p{4cm} | p{.5cm} | p{7cm}|} 
\hline
\rowcolor{blue}
\multicolumn{2}{|l|}{\color{white}\bfseries{Nom}} & \multicolumn{3}{l|}{\color{white}\bfseries{#1}}\\
\hline
\multicolumn{2}{|l|}{\bfseries{Acteurs concernés}} & \multicolumn{3}{m{10.5cm}|}{#2}\\
\hline
\multicolumn{2}{|l|}{\bfseries{Description}} & \multicolumn{3}{m{10.5cm}|}{#3}\\
\hline
\multicolumn{2}{|l|}{\bfseries{Préconditions}} & \multicolumn{3}{m{10.5cm}|}{#4}\\
\hline
\multicolumn{2}{|l|}{\bfseries{Evénements déclenchants}} & \multicolumn{3}{m{10.5cm}|}{#5}\\
\hline
\multicolumn{2}{|l|}{\bfseries{Conditions d'arrêt}} & \multicolumn{3}{m{10.5cm}|}{#6}\\
\hline
\rowcolor{gray}
\multicolumn{5}{|c|}{\bfseries{Description du flot d'événements principal}}\\
\hline
\rowcolor{gray}
\multicolumn{3}{|c|}{\bfseries{Acteur(s)}} & \multicolumn{2}{c|}{\bfseries{Système}}\\
\hline
\multicolumn{3}{|p{7.5cm}|}{#7} & \multicolumn{2}{p{7.5cm}|}{#8}\\
\hline
\multicolumn{2}{|l}{\bfseries{Flots d'exceptions}} & \multicolumn{3}{|p{11.5cm}|}{#9}\\
\hline
\end{tabular}
\\
}

\definecolor{gris}{rgb}{0.95, 0.95, 0.95}

\begin{document}
\pagestyle{fancy}

\vspace*{5cm}
\begin{center}\textbf{\Huge{\hmwkDocName}}\end{center}
\vspace*{7cm}
	
\begin{center}
\fcolorbox{black}{gris}{
\begin{minipage}{10cm}
\begin{tabularx}{10cm}{lXl}
	\bfseries{Version} & & \version\\
	& & \\
	\bfseries{Date} & & \docDate\\
	& & \\
	\bfseries{Rédigé par} & & \hmwkAuthorName \\
	& & \\
	\bfseries{Relu par} & & \checked \\
	& & \\
	\bfseries{Approuvé par} & & \approved \\
	& & \\
\end{tabularx}
\end{minipage}
}
\end{center}

\newpage

%Tableau de mises à jour
\vspace*{1cm}
\begin{center}
\textbf{\huge{MISES À JOUR}}\\
\vspace*{3cm}
	\begin{tabularx}{16cm}{|c|c|X|}
	\hline
	\bfseries{Version} & \bfseries{Date} & \bfseries{Modifications réalisées}\\
	\hline
	X & X & X\\
	\hline
	\end{tabularx}
\end{center}

%La table des matières
\clearpage
\tableofcontents
\clearpage

% OBJET
\section{Objet}


% DOCUMENTS APPLICABLES ET DE RÉFÉRENCE
\section{Documents applicables et de référence}
\begin{itemize}
\item x
\end{itemize}

% TERMINOLOGIE
\section{Terminologie et sigles utilisés}
\begin{itemize}
\item \textbf{x} : xx
\end{itemize}

% EXIGENCES FONCTIONNELLES
\section{Exigences fonctionnelles}
\subsection{Présentation de la mission du produit logiciel}
\begin{itemize}
\item [EF.x] : xxx ;
\end{itemize}

\begin{center}
\begin{tabular}{|l|p{6cm}|p{6cm}|l|}
\hline
\bfseries{Id} & \bfseries{Intitulé} & \bfseries{Acteur(s)} & \bfseries{Priorité}\\
\hline
UC.X &  &  & \\
\hline
\end{tabular}
\end{center}


\begin{itemize}
\item [RG.X] xxx
\end{itemize}
\subsection{UC X : xxx}

\fiche
	{} %nom
	{} %acteurs concernés
	{} %description
	{} %préconditions
	{} % événements déclenchants
	{} %conditions d'arret
	{\begin{itemize}  %flot d'événements (acteurs)
		\item [1.] xx
	 \end{itemize}
	} 
	{\begin{itemize}  %flot d'événements (systeme)
		\item []
		\item [2.] xxx
	\end{itemize}
	 }
	{} %Flots d'exceptions (RG)


\section{Diagrammes de cas d'utilisation}

\subsection{UC X}

\begin{tikzpicture}
\end{tikzpicture}

\section{Exigences opérationnelles}
\begin{enumerate}
\item [EO.X] : xxx
\end{enumerate}

\section{Exigences d'interface}
\begin{enumerate}
\item [EI.X] : xxx
\end{enumerate}

\section{Exigences de qualité}
\begin{enumerate}
\item [EQ.X] : xxx
\end{enumerate}

\section{Exigences de réalisation}
\begin{enumerate}
\item [ER.X] : xxx
\end{enumerate}


\end{document}