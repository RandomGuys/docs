\documentclass{article}
\usepackage[utf8]{inputenc}
\usepackage[T1]{fontenc}
\usepackage[francais]{babel}
\usepackage[top=1.5cm, bottom=2cm, left=1.5cm, right=1.5cm, includeheadfoot]{geometry} 
\usepackage{lmodern}
\usepackage{tikz}        
\usepackage{fancyhdr} 
\usepackage{lastpage} 
\usepackage{extramarks}
\usepackage{graphicx} 
\usepackage{tabularx, longtable}
\usepackage{color, colortbl}
\usepackage{amsmath}
\usepackage{mathtools, bm}
\usepackage{amssymb, bm}
\usepackage[toc,page]{appendix} 
\usepackage{pgfplots}
\usepackage{eurosym}
\usepackage{rotating}
\usepackage{array}
\usepackage{cmbright}
\usepackage{hyperref}
	\hypersetup{
		hyperfigures = true,
		colorlinks = true,
		linkcolor = black,
		urlcolor = blue
	}
\usepackage[french]{algorithm2e}
\usepackage{cmbright}
\usepackage{listings}

\usepgflibrary{arrows}

\linespread{1.1}
\setlength{\headheight}{40pt}

\title{\vfill \textbf{Analyse de failles directes et indirectes à OpenSSL}}
\author{Mathieu Latimier et Claire Smets}
\date{\today \vfill}

\begin{document}




\maketitle




\newpage




\tableofcontents




\newpage




\section{Introduction}

	Sur ce document, nous nous focaliserons sur les failles et les correctifs 
	à appliquer sur les primitives cryptographiques – selon les systèmes
	utilisés.
	Dans l'ordre : la génération de l'aléa, la génération des clés, 
	l'échange des clés, les signatures et vérifications, l'authentification, la génération des masques et les protocoles.
	J’émets également des doutes sur le code utilisé par OpenSSL, notamment sur les 
	appels de PRNG comme /dev/urandom.\\
	
	La partie "Forces et faiblesses des différents PRNG" a été rédigé
	conjointement avec Pascal Edouard et Julien Legras.\\
	
	Enfin, une partie est consacrée sur la possibilité d'une back-door par 
	la NSA dans les outils de chiffrement (dont OpenSSL) et une corruption au 
	sein des normes et des standards (dont le NIST).\\
	
	Vous trouverez l'ensemble de nos sources en Annexe, si elles ne sont pas 
	déjà citées en cours de document. Les sources proviennent de l'ensemble de
	l'équipe.

	\newpage











\section{Failles générales d'OpenSSL}


	\subsection{Common Vulnerability \& Exposure}
	
	Pour connaître l'ensemble des vulnérabilités et des expositions d'OpenSSL 
	il est intéressant d'étudier les CVE (Common Vulnerability and Exposure).
	
	Voici le lien de toutes les CVE concernant OpenSSL : \\

	\textbf{Source} :
	\href{http://www.cvedetails.com/vendor/217/Openssl.html}
	{www.cvedetails.com/vendor/217/	Openssl.html}\\

	Vous pouvez également parcourir la base de données des vulnérabilités 
	du NIST. Voici la recherche pour "OpenSSL" : \\

	\textbf{Source} :
	\href{http://web.nvd.nist.gov/view/vuln/search-results?query=openssl\&search\_type=all\&cves=on}
	{web.nvd.nist.gov/view/vuln/search-results?query=openssl\&search\_type=all\&cves=on}\\

	On retrouve bien toutes les CVE, cependant l'affichage est moins 	
	esthétique que le précédent lien. Il peut toutefois être un bon complément
	pour la recherche d'informations sur une vulnérabilité particulière\\

	Pour jeter un coup d'oeil sur le statut général des CVE OpenSSL :\\
	\href{https://supportcenter.checkpoint.com/supportcenter/portal?eventSubmit\_doGoviewsolutiondetails\&solutionid=sk92447}
	{supportcenter.checkpoint.com/supportcenter/portal?eventSubmit\_doGoviewsolutiondetails\&solutionid=sk92447}\\

	\subsection{Common Weakness Enumeration}

	Pour ce qui est des failles logicielles (qui ne concerne OpenSSL qu'au 
	niveau de l'application où il est installé), il y a le site CWE (Common 
	Weakness Enumeration). Sur l'onglet de recherche entrez "OpenSSL" :\\

	\textbf{Source} :
	\href{http://cwe.mitre.org/find/index.html}{cwe.mitre.org/find/index.html}\\


	\subsection{Exemples de failles}
	
	Voici quelques failles récentes d'OpenSSL:\\
	
	\textbf{Le 07 janvier 2014}, un déférencement de pointeur NULL via ssl3\_take\_mac() 
	est la cause de DoS (Deny of Service) : \\
	
	Les produits concernés sont Debian, Fedora, FreeBSD, Copssh, \textbf{OpenSSL}, 
	openSUSE, pfSense, RHEL, Slackware.\\
	
	La bibliothèque OpenSSL implémente SSL/TLS, pour les clients et les serveurs.\\

	La fonction ssl3\_take\_mac() du fichier ssl/s3\_both.c est utilisée par les clients 
	pour calculer le Finished MAC. Cependant, la fonction ssl3\_take\_mac() ne vérifie 
	pas si un pointeur est NULL, avant de l'utiliser.\\

	Un serveur TLS illicite peut donc envoyer un handshake invalide vers le client 
	OpenSSL, pour déréférencer un pointeur NULL, afin de mener un déni de service.\\

	\textbf{Le 06 janvier 2014}, plusieurs vulnérabilités ont étés détectées sur OpenSSL.	
	Elles permettraient à un attaquant distant de provoquer des DoS.\\
	
	Versions affectées : antérieurs à OpenSSL 1.0.0l et 1.0.0f\\
	
	\textbf{Source} :\\ 
	\href{http://www.cert.ssi.gouv.fr/site/CERTA-2014-AVI-003/CERTA-2014-AVI-003.html}
	{www.cert.ssi.gouv.fr/site/CERTA-2014-AVI-003/CERTA-2014-AVI-003.html}\\

	\textbf{Source} : \\
	\href{http://vigilance.fr/vulnerabilite/OpenSSL-dereferencement-de-pointeur-NULL-via-ssl3-take-mac-14029}
	{vigilance.fr/vulnerabilite/OpenSSL-dereferencement-de-pointeur-NULL-via-ssl3-take-mac-14029}\\
	
	\textbf{Le 02 janvier 2014}, une faille est détecté sur le protocole DTLS d'OpenSSL. \\
	Les produits concernés sont : Debian, Fedora, FreeBSD, Copssh, MBS, 
	\textbf{OpenSSL}, openSUSE, pfSense, Puppet, RHEL, Slackware.\\

	Le protocole DTLS (Datagram Transport Layer Security), basé sur TLS, fournit 
	une couche cryptographique au-dessus du protocole UDP.\\

	Cependant, la fonction dtls1\_hm\_fragment\_new() du fichier ssl/d1\_both.c 
	ne gère pas correctement l'état de retransmission.\\

	Un attaquant, placé en Man-in-the-middle, peut donc forcer l'utilisation 
	d'un contexte DTLS invalide dans OpenSSL, afin d'obtenir des informations sensibles.\\

	\textbf{Source} : \\
	\href{http://vigilance.fr/vulnerabilite/OpenSSL-obtention-d-information-via-DTLS-14007}
	{vigilance.fr/vulnerabilite/OpenSSL-obtention-d-information-via-DTLS-14007}\\

	\textbf{Le 12 décembre 2013}, TLS 1.2 est vulnérable à une attaque de type DoS.\\
	
	Produits concernés : Debian, Fedora, FreeBSD, Copssh, \textbf{OpenSSL}, 
	openSUSE, pfSense, RHEL, Slackware.\\

	La bibliothèque OpenSSL supporte les versions 1.0 à 1.2 de TLS.\\

	La fonction ssl\_get\_algorithm2() du fichier ssl/s3\_lib.c obtient la 
	version de la session en cours. Cependant, cette fonction utilise 
	une structure qui n'est pas à jour parfois. Une erreur interne se produit alors.\\

	Un attaquant peut donc employer TLS 1.2 avec une application liée à OpenSSL, 
	afin de mener un déni de service.\\
	
	\textbf{Source} : \\
	\href{http://vigilance.fr/vulnerabilite/OpenSSL-deni-de-service-via-TLS-1-2-13978}
	{vigilance.fr/vulnerabilite/OpenSSL-deni-de-service-via-TLS-1-2-13978}\\

	\textbf{Le 5 février 2013}, deux failles de type DoS sur OpenSSL ont étés publiés.\\
	La première est possible en combinant CBD et AES-NI.\\
	Les produits concernés sont : HP-UX, Tivoli Workload Scheduler, 
	\textbf{OpenSSL}, openSUSE, Slackware.\\
	
	Depuis 2008, certains processeurs de la famille x86 implémentent les 
	instructions assembleur AES-NI. Elles permettent de demander au processeur 
	d'effectuer des calculs AES en une seule instruction.\\

	Lorsqu'une application liée à OpenSSL s'exécute sur un processeur avec AES-NI, 
	un attaquant peut stopper les sessions TLS en mode CBC.\\

	Les détails techniques ne sont pas connus. L'erreur pourrait être située 
	dans la fonction aesni\_cbc\_hmac\_sha1\_cipher() du fichier 
	crypto/evp/e\_aes\_cbc\_hmac\_sha1.c.\\

	\textbf{Source} : \\
	\href{http://vigilance.fr/vulnerabilite/OpenSSL-deni-de-service-via-CBC-et-AES-NI-12377}
	{vigilance.fr/vulnerabilite/OpenSSL-deni-de-service-via-CBC-et-AES-NI-12377}\\

	La seconde provient de OCSP.\\
	
	Produits concernés : Debian, BIG-IP Appliance, Fedora, FreeBSD, HP-UX, AIX, 
	Tivoli Workload Scheduler, Juniper J-Series, JUNOS, MBS, MES, McAfee Email 
	and Web Security, \textbf{OpenSSL}, openSUSE, Solaris, pfSense, RHEL, 
	JBoss Enterprise, Slackware, ESX, ESXi, vCenter, VMware vSphere Hypervisor.\\
	
	L'extension OCSP (Online Certificate Status Protocol) vérifie la validité 
	des certificats.\\

	La fonction OCSP\_basic\_verify() du fichier crypto/ocsp/ocsp\_vfy.c décode 
	la réponse OCSP reçue. Cependant, si la clé est vide, un pointeur NULL est 
	déréférencé.\\

	Un attaquant peut donc mettre en place un serveur OCSP illicite, 
	afin de stopper les applications OpenSSL qui s'y connectent.\\
	
	\textbf{Source} : \\
	\href{http://vigilance.fr/vulnerabilite/OpenSSL-deni-de-service-via-OCSP-12378}
	{vigilance.fr/vulnerabilite/OpenSSL-deni-de-service-via-OCSP-12378}\\

	Nous remarquons que tout les mois plusieurs failles sur OpenSSL apparaissent, et
	majoritairement des DoS.\\ 
	
	Nous pourrons donc nous servir des précédentes failles pour notre audit, afin
	de détecter d'éventuelles failles du même type.


	\newpage











\section{Failles au niveau du générateur de l'aléa}

	\subsection{Debian 4.0 et OpenSSL 0.9.8}
	
	Il n'y a à priori aucune faille sur les générateurs d'aléas dans OpenSSL.\\
	Cependant, l'utilisation de générateurs d'aléas provenant de systèmes 
	d'exploitations, peuvent engendrer de fortes pertes d'entropie,
	voir la rendre totalement nulle.\\

	Le 13 Mai 2008, Luciano Bello découvert une faille critique du paquet d'OpenSSL 
	sur les systèmes Debian. Un mainteneur Debian souhaitant corriger quelques 
	bugs aurait malencontreusement supprimé une grosse source d'entropie lors 
	de la génération des clés. \\
	Il ne restait plus que le PID comme source d'entropie!\\
	Comme celui-ci ne pouvait dépasser 32.768 (qui le PID maximal
	atteignable), l'espace des clés a été restreint à 264.148 clés
	distinctes.\\

	\textbf{Source} : \\
	\href{http://linuxfr.org/news/d\%C3\%A9couverte-dune-faille-de-s\%C3\%	A9curit\%C3\%A9-critique-dans-openssl-de-deb}
	{linuxfr.org/news/d\%C3\%	A9couverte-dune-faille-de-s\%C3\%A9curit\%C3\%A9-critique-dans-openssl-de-deb} \\

	Analysons plus en détail cette faille. Elle se situe au niveau de 
	la fonction \textbf{md\_rand.c}.
	La ligne \textbf{MD\_Update(\&m, buf, j);} a été commentée.
	La conséquence est le blocage de la graine (seed) que l'on 
	passe ensuite au PRNG.\\
	
	Cette ligne a été commentée par erreur en voulant corriger un
	avertissement soulevé par le compilateur Valgrind sur une valeur
	non initialisé.\\
	
	Le 14 Mai 2008, Steinar H. Gunderson démontre simplement comment
	en connaissant le secret k d'une signature, on peut retrouver
	la clé privée d'un certificat immédiatement.\\
	
	Ce secret k étant généré avec un PRNG prévisible, on peut stocker
	deux signatures utilisant le même k, où le prédire directement.\\
	
	Une signature DSA consiste en deux nombres r et s tels que :\\
	r = (g$^k$ [p]) [q]\\
	s = (k$^-1$ * (H(m) + x * r)) [q]\\
	
	La clé publique = (p, q, g).\\
	Le message en clair = m, et H(m) est le fingerprint de m connu.\\
	
	Attaque \no 1 : En connaissant k	\\
	s * k [q] = (H(m) + x*r) [q]\\
	<==> (s * k – H(m)) [q] = x*r [q]\\
	<==> ((s*k – H(m))*r$^-1$)[q] = x\\
	<==> (s*k – H(m))*r$^-1$ = x\\
	
	Attaque \no 2 : Deux messages possèdent le même k\\
	s1 = (k-1 (H(m1) + xr)) [q]\\
	s2 = (k-1 (H(m2) + xr)) [q]\\
	<==> s1 - s2 = (k-1 (H(m1) - H(m2)) [q]\\
	<==> (s1 - s2)(H(m1) - H(m2))-1 = k-1 [q]\\
	<==> On connaît k ==> Attaque 1\\
	
	\textbf{Source} : \\
	\href{http://plog.sesse.net/blog/tech/2008-05-14-17-21_some_maths.html}
	{plog.sesse.net/blog/tech/2008-05-14-17-21\_some\_maths.html}	
	
	Pour savoir si une clé SSL, SSH, DNSSEC ou OpenVPN est affectée, un 	
	détecteur de données de clés faibles est fourni par l'équipe 
	Security de Debian : \\
	\href{http://security.debian.org/project/extra/dowkd/dowkd.pl.gz}
	{security.debian.org/project/extra/dowkd/dowkd.pl.gz} \\

	Il y a également un logiciel plus simple d'utilisation toujours développé 
	par l'équipe Debian Security - openssl-vulnkey - que vous pouvez 
	télécharger à l'adresse ci-dessous :\\
	\href{http://packages.debian.org/fr/sid/openssl-blacklist}
	{packages.debian.org/fr/sid/openssl-blacklist}\\
	
	Pour voir l'avertissement de sécurité de l'équipe Debian : \\
	
	\textbf{Source :} \\
	\href{https://www.debian.org/security/2008/dsa-1571}
	{www.debian.org/security/2008/dsa-1571}

	\subsection{LinuxMintDebianEdition et Android}

	On pensait alors que cette faille ne surviendrait plus, que c'était une 
	erreur farfelue (qui a tout de même durée plus de deux ans sur l'un des 
	systèmes les plus en vogue - L'erreur datant de Septembre 2006). \\
	
	Et récemment, en Août 2013 précisément, un patch de sécurité pour les 
	systèmes Android utilisant la version LinuxMintDebianEdition/OpenSSL, 
	dévoile une réparation du générateur de nombres pseudo-aléatoire (PRNG) 
	qui ne donnait pas suffisamment d'entropie. \\
	
	Le patch indique que le PRNG de cette version d'OpenSSL utilise dorénavant 
	une combinaison de données plus ou moins prévisibles associées à 
	l'entropie générées par /dev/urandom. \\
	Mais sachant que le PRNG d'OpenSSL utilise lui-même /dev/urandom, on a du 
	mal à comprendre pourquoi en rajouter davantage.\\

	Eric Wong et Martin Boßlet apporte la solution sur leur site, l'erreur 
	provient d'un bug "à la Debian", une simple ligne diffère de la version 
	officielle d'OpenSSL (utilisant SecureRandom) à celle de OpenSSL::Random 
	ce situant dans la fonction ssleay\_rand\_bytes. \\

	La cause est là même que celle de Debian, un patch de sécurité atteint la 
	source d'entropie du PRNG. Alors que tout semblait être rentré dans 
	l'ordre, un résidu de cette erreur reste dans cette version. Les 
	développeurs d'OpenSSL assure que ça n'a pas d'impact (ou alors très peu) 
	sur la sécurité globale. 
	Mais à cause de la mémoire non initialisé des systèmes Android, la source 
	d'entropie ne nous permet pas de générer des nombres non-prédictibles. \\

	La conséquence n'est pas aussi lourde que celle de Debian, tout d'abord 
	parce que le système Android est rarement utilisé pour du 
	chiffrement de données sensible, et une attaque par prédiction bien que
	plus rapide qu'une attaque par brute-force, reste infaisable. 
	Mais l'erreur est quand même là. \\

	\textbf{Sources} : 
	\begin{itemize}
		\item \href{http://emboss.github.io/blog/2013/08/21/openssl-prng-is-not-really-fork-safe/}
		{emboss.github.io/blog/2013/08/21/openssl-prng-is-not-really-fork-safe/}
		\item \href{http://www.nds.rub.de/media/nds/veroeffentlichungen/2013/03/25/paper\_2.pdf}
		{www.nds.rub.de/media/nds/veroeffentlichungen/2013/03/25/paper\_2.pdf}
		\item \href{http://android-developers.blogspot.de/2013/08/some-securerandom-thoughts.html}
		{android-developers.blogspot.de/2013/08/some-securerandom-thoughts.html}\\
	\end{itemize}

	\subsection{NetBSD 6.0 et OpenSSH}

	Autre faille du même genre sur les systèmes NetBSD 6.0 (Mais sur OpenSSH 
	cette fois-ci) et datant de Mars 2013 : \\

	C'est le syndrome OpenSSH de Debian qui frappe une nouvelle fois.\\
	Du fait d'une parenthèse mal placée dans le code du fichier 
	/src/sys/kern/subr\_cprng.c, il s'avère que le générateur pseudo-aléatoire 
	de NetBSD 6.0 est bien moins solide que ce qui était attendu.\\
	C'est une manière polie de dire que sa sortie n'est pas assez aléatoire et 
	qu'il faut d'urgence changer les clés SSH qui ont été générées avec ce noyau !\\

	\textbf{L'alerte de sécurité} : \href{http://ftp.netbsd.org/pub/NetBSD/security/advisories/NetBSD-SA2013-003.txt.asc}
	{ftp.netbsd.org/pub/NetBSD/security/advisories/NetBSD-SA2013-003.txt.asc}\\
	
	\textbf{Un article de h-online} : \href{http://www.h-online.com/open/news/item/Weak-keys-in-NetBSD-1829336.html}
	{www.h-online.com/open/news/item/Weak-keys-in-NetBSD-1829336.html}\\

	Le problème est corrigé dans NetBSD Current et le fix sera disponible dans 
	la furture version NetBSD 6.1.
	Il est probable que les dégâts seront bien moins étendus que lors de l'affaire 
	OpenSSH dans Debian car les systèmes NetBSD 6.0 ne sont sans doute pas très 
	fréquents.\\
	
	Toutes les clefs générées en utilisant /dev/urandom sont vulnérables.\\

	\textbf{Analysons le diff} : \\
	\href{http://cvsweb.netbsd.org/bsdweb.cgi/src/sys/kern/subr\_cprng.c.diff?r1=1.14\&r2=1.15\&only\_with\_tag=MAIN\&f=h)}
	{cvsweb.netbsd.org/bsdweb.cgi/src/sys/kern/subr\_cprng.c.diff?r1=1.14\&r2=1.15\&only\_with\_tag=MAIN\&f=h)}. \\ 
	
	Pour la partie concernant la ligne 183: rnd\_extract\_data
	(key + r, sizeof(key - r), RND\_EXTRACT\_ANY);\\
	
	Le deuxième paramètre devrait être sizeof(key) - r.\\
	
	Il semble que dans la version 1.15, cette appel a été déplacé (et corrigé
	rapidement) vers la nouvelle fonction cprng\_entropy\_try.

	\textbf{Source} : \\
	\href{http://linuxfr.org/users/patrick\_g/journaux/faille-de-securite-critique-dans-le-generateur-pseudo-aleatoire-de-netbsd-6-0}
	{linuxfr.org/users/patrick\_g/journaux/faille-de-securite-critique-dans-le-generateur-pseudo-aleatoire-de-netbsd-6-0}.\\
	
	\subsection{Analyse de la faille Debian avec OpenSSL-Vulnkey}
	
	Finalement, nous avons décidé de tester le nombre de certificats vulnérables 
	causés 	par le bug OpenSSL de Debian (qui reste le plus populaire), et
	connaissant la blacklist des clés privés. \\
	
	Les résultats nous montrent que sur 500.000 certificats récupérés, au moins
	\footnote{Le logiciel ne prend pas en compte les clés $\leq$ à 512 bits et 
	celles $\geq$ à 4096 bits, et ne prend en compte que les certificats RSA} 
	769 sont vulnérables.\\

	Vous pouvez trouver nos scripts parcourant un fichier contenant un 
	certificat sur chaque ligne, ou un dossier contenant des certificats sous 
	forme de fichiers PEM et nos résultats, dans le dossier consacré à l'audit des clés
	cryptographiques.\\

	Le format de nos résultats est : \\
	
	\textbf{COMPROMISED:} \textit{<haché\_du\_certificat>} \textit{<nom
	\_fichier\_corrompu (sous forme d'adresse IP)}\\

	Évidemment, nous ne mettons pas ces résultats sur le net puisqu'il indique 
	très clairement les adresses IP contenant le certificat friable, et sa clé
	privée (que l'on peut facilement retrouvé parmi la courte blacklist).\\

	Pour information, parmi les entreprises vulnérables nous trouvons les géants
	IBM et CISCO.

	\newpage







\section{Forces et faiblesses des différents PRNG}

	Nos machines utilise ce qu'on appel des PRNG (Pseudo Random Number Generator), 
	ce sont des algorithmes qui génèrent une séquence de nombre s'apparentant
	à de l'aléatoire. En réalité rien est aléatoire car tout est déterminé par
	des valeurs initiales (État du PRNG) et des contextes d'utilisation.\\
	
	Un bon PRNG se doit d'avoir une très forte entropie (proche de un), afin
	d'éviter de délivrer de l'information.

	Comme l'entropie est fourni majoritairement (si ce n'est totalement) par 
	l'OS, il est donc nécessaire de détailler les PRNG les plus utilisés
	(Surtout par les systèmes Linux et BSD - qui sont les ceux qui générent le
	plus de certificats SSL).\\

	Nous nous basons sur la RFC 4086 : Randomness requirements for security
	pour le choix des PRNG selon les différents	systèmes.
	
	\textbf{Source} :
	\href{http://www.ietf.org/rfc/rfc4086.txt}{www.ietf.org/rfc/rfc4086.txt}




	\subsection{/dev/urandom et /dev/random sous Linux}
	
		Sous Linux, un pool est initialisé avec 512 octets, auquel on ajoute
		le temps émis par un évènement et son état parmis : 
		\begin{itemize}
			\item Les interruptions clavier - heure et code d'interruption
			\item Les interruptions de disques - heure de lecture ou écriture
			\item Les mouvements de souris - heure et position\\
		\end{itemize}
	
		Quand des octets aléatoires sont demandés, la pool est haché avec SHA-1 
		(20 octets). S’il est demandé plus que 20 octets, le haché est mélangé 
		dans la pool pour rehacher la pool ensuite etc. À chaque fois que l’on 
		prend des octets dans la pool, l’entropie estimée est décrémentée.\\

		Pour assurer un niveau minimum d’entropie au démarrage, la pool est 
		écrite dans un fichier à l’extinction de la machine.\\

		/dev/urandom fonctionne selon le même principe sauf qu’il n’attend pas 
		qu’il y ait assez d’entropie pour donner de l’aléatoire. Il convient 
		pour une génération de clefs de session.\\

		Pour générer des clefs cryptographiques de longue durée, il est 
		recommandé d’utiliser /dev/random pour assurer un niveau minimum 
		d’entropie.\\
		
		En effet, sur un serveur sans souris ni clavier, définir l'entropie avec
		/dev/urandom est très risqué. On recommande donc l'utilisation de 
		/dev/random lors de l'audit OpenSSL sur les versions Linux.\\

		/dev/random utilise une pool d’entropie de 4096 bits (512 octets) génère 
		de l’aléa et s’arrête lorsqu’il n’y a plus assez d’entropie et attend 
		que la pool se remplisse à nouveau.\\
		
		Si vous souhaitez connaître l'entropie disponible, la commande est : \\
		cat /proc/sys/kernel/random/entropy\_avail\\

		Désormais, la taille de la pool est hardcodée dans le noyau Linux 
		(/drivers/char/random.c:275)\\
		
		Linux offre également la possibilité de récupérer de l’aléa depuis un 
		RNG matériel avec la fonction get\_random\_bytes\_arch\\

		\textbf{Source} :
		\href{https://wiki.archlinux.org/index.php/Random\_Number\_Generation}
		{wiki.archlinux.org/index.php/Random\_Number\_Generation}
	
		Un patch est également disponible afin de générer de l'aléa avec un 
		débit de 100kB/s. L'entropie est récupérée par le CPU timing jitter.\\
		
		\textbf{Source} :
		\href{http://lkml.iu.edu//hypermail/linux/kernel/1302.1/00479.html}
		{lkml.iu.edu//hypermail/linux/kernel/1302.1/00479.html}
		
		En conclusion, /dev/random doit être utilisé pour une haute qualité
		d'entropie (i.e. haute sécurité de chiffrement, one-time pad).\\
		Tandis que /dev/urandom doit être utilisé pour des applications non
		sensibles à des attaques cryptographiques (i.e. jeu en temps réel),
		car elle génère plus d'entropie que /dev/random sur un temps donné, 
		mais s'arrêtera même si il n'a pas récolté suffisamment d'entropie. 
	
	
	
	
	\subsection{/dev/random sous FreeBSD et /dev/arandom sous OpenBSD}
	
		Il faut faire attention au faux ami, le /dev/random du FreeBSD n'est pas 
		le même que celui de Linux.
		En fait, il est semblable au /dev/urandom de Linux, et est donc tout 
		autant proscrit lors de notre audit.\\
		
		Même principe avec le /dev/arandom de OpenBSD, qui a également une 
		entropie faible pour du chiffrement cryptographique sûr. Il se base
		en fait sur un algorithme modifié du RC4 nommé ARC4 (Alleged RC4) pour 
		générer des données aléatoires.\\
		Pour rappel, RC4 était un projet commercial de la RSA Security, et un 
		hacker anonyme a publié un code identique, devenu légitime, identifié par
		ARC4.\\
		De nos jours, il est fortement conseillé de ne plus utiliser RC4 car le 
		flux de données aléatoire n’est pas vraiment aléatoire et il existe des 
		attaques qui prédisent la sortie de l’algorithme (Attaque de Fluhrer, 
		Mantin et Shamir).\\
		Sur plusieurs de nos sources (plus anciennes), il est recommandé
		d'utiliser /dev/arandom pour sa rapidité (71 Mb/s) et sa bonne source
		d'entropie. Ce n'est plus vraiment le cas aujourd'hui.\\

		\textbf{Source} :
		\href{http://www.cs.rit.edu/~axl6334/crypto/Report.pdf}
		{www.cs.rit.edu/~axl6334/crypto/Report.pdf}
		
		
		
	
	\subsection{CryptGenRandom sous Windows}
	
		Du coté de Microsoft, il recommande aux utilisateurs de Windows 
		d’utiliser CryptGenRandom, qui est un appel système de génération d’un 
		nombre pseudo-aléatoire. La génération est réalisée par une librairie 
		cryptographique (Cryptographic service provider library). Celui ci gère 
		un pointeur vers un buffer en lui fournissant de l’entropie afin de 
		générer un nombre pseudo aléatoire en retour avec en plus, le nombre 
		d’octet d’aléatoire désiré.\\

		\lstset{language=Java}
		\begin{lstlisting}
		BOOL WINAPI CryptGenRandom(
			_In_     HCRYPTPROV hProv,
			_In_     DWORD dwLen,
			_Inout_  BYTE *pbBuffer
		);
		\end{lstlisting}
	
		Le service provider sauvegarde une variable d’état d’un sel pour chaque 
		utilisateur. Lorsque CryptGenRandom est appelé, celui ci est combiné 
		avec un nombre aléatoire généré par la librairie en plus de différentes 
		données systèmes et utilisateurs telles que : \\
		\begin{itemize}
		\item l’ID du processus
		\item l'ID du thread
		\item l'horloge système
		\item l'heure système
		\item l'état de la mémoire
		\item l’espace de disque disponible du cluster
		\item le haché du block d'environnement mémoire de l’utilisateur\\
		\end{itemize} 
		
		Le tout est envoyé à la fonction de hachage SHA-1 et le nombre en sortie 
		est utiliser comme sel pour une clef RC4. \\
		
		Cette clef est enfin utilisé pour produire des données pseudo aléatoire 
		et mettre à jour la variable d’état du sel de l’utilisateur. 

	
	
	\subsection{Autres systèmes}
	
		Nous avons également d'autres RNG comme srandom, prandom, wrandom, ici 
		sur MirOS BSD : \\
		\href{https://www.mirbsd.org/htman/i386/man4/arandom.htm}
		{www.mirbsd.org/htman/i386/man4/arandom.htm}\\
		
		/dev/srandom est simple et lent, il n'est pas recommandé de l'utilisé.\\
	
		Certains systèmes ne disposant pas de /dev/*random, il est alors possible
		d'utiliser l'EGD (Entropy Gathering Daemon).\\
		
		\textbf{Source} : \\
		\href{http://egd.sourceforge.net/}{egd.sourceforge.net/}
		
		Il faut pour cela utiliser les fonctions OpenSSL RAND\_egd, RAND\_egd\_bytes
		et RAND\_query\_egd\_bytes. \\
		
		L'EGD est également utilisé par GPG, et peut être utilisé comme seed.

		\newpage








\section{Failles au niveau de la génération des clés}

	\subsection{Failles concernant Diffie-Hellman}
	
	Lorsque nous avons étudiés le code de Diffie-Hellman dans OpenSSL, nous nous
	sommes penchés sur un choix plutôt étrange . \\
	La valeur du générateur est toujours fixé à 2 ou à 5. \\

	Le générateur de Diffie-Hellman n'étant pas une racine primitive dans 
	$\mathbb{Z}/\mathbb{Z}_p$, les conséquences sont :
	\begin{itemize}
	\item L'espace des clés possibles est fortement réduit (Si g=2 => espace 
	divisé par deux)
	\item Deux clés privées distinctes pourront avoir une clé publique commune
	\item La méthode de cryptanalyse Baby-step Giant-step peut s'en trouver 
	facilité.\\
	\end{itemize}

	Évidemment, ce choix n'est pas une faille en soit, il n'est juste pas 
	optimal et résulte d'un bon compromis entre vitesse et sécurité.\\
	
	Pour une sécurité optimale, il est conseillé de choisir un générateur qui 
	soit une racine primitive, pour être certain que personne ne puisse 
	signer, déchiffrer des messages à votre place!\\

	Voici l'algorithme de génération de g (d'après la RFC 2631 - 1999 / dérivé 
	de [FIPS-186]) : \\
	\begin{itemize}
	\item 1- Soit j = (p - 1)/q.
	\item 2- Choisir h $\in \mathbb{N}$, tel que 1 < h < p - 1
	\item 3- Calculer g = $h^j$ mod p
	\item 4- Si g = 1 recommencer l'étape 2\\
	\end{itemize}

	Mais depuis 2006, on peut lire comme recommandation dans la RFC 4419 (Pour 
	une utilisation SSH) : \\
	" It is recommended to use 2 as generator, because it improves
	efficiency in multiplication performance.  It is usable even when it
	is not a primitive root, as it still covers half of the space of
	possible residues. "\\
	
	
	\textbf{Sources} : \\
	\href{http://tools.ietf.org/html/rfc2631}{tools.ietf.org/html/rfc2631}\\
	\href{http://tools.ietf.org/html/rfc4419}{tools.ietf.org/html/rfc4419}\\

	Une faille plus grave concerne le mode FIPS (Federal Information 
	Processing Standard) d'OpenSSL.
	Qui peut être compilé avec la commande "./config fipscanisterbuild". \\

	En effet, un attaquant situé entre le client et le serveur, et connaissant
	la clé secrète du serveur, peut déchiffrer une session SSL/TLS. \\
	
	L'algorithme EDH/DHE (Diffie-Hellman Ephémère) permet de calculer 
	une nouvelle clé connue uniquement du client et du serveur, donc l'attaquant 
	intermédiaire ne peut plus déchiffrer la session.
	Cependant, en mode FIPS, OpenSSL ne rejette pas les paramètres P/Q faibles 
	pour EDH/DHE.

	Lorsque OpenSSL est compilé en mode FIPS, un attaquant en Man-in-the-middle 
	peut donc forcer la génération d'un secret Diffie Hellman prédictible.

	\textbf{Source} : \\
	\href{http://vigilance.fr/vulnerabilite/OpenSSL-Man-in-the-middle-FIPS-Diffie-Hellman-10585}
	{vigilance.fr/vulnerabilite/OpenSSL-Man-in-the-middle-FIPS-Diffie-Hellman-10585}

	\textbf{Date} : Avril 2011

	\newpage
	
	
	\subsection{Faille concernant RSA-OAEP}
	
	RSA-OAEP peut être soumis à une attaque nommée "Mangers Attack" selon son
	implantation. OpenSSL semble être vulnérable à une attaque de ce type, à
	base de "prédictions".\\
	La vulnérabilité semble être très récente puisqu'elle fonctionne sous
	OpenSSL 1.0.0.\\
	
	La Technische Universität Darmstadt (Allemagne) apporte des contre-mesures,
	et note qu'il existe toutefois des cas où l'algorithme est plus résistant.\\
	
	\textbf{Source :} \\
	\href{http://www.cdc.informatik.tu-darmstadt.de/reports/reports/mangers\_attack\_revisited.pdf}
	{www.cdc.informatik.tu-darmstadt.de/reports/reports/mangers\_attack\_revisited.pdf}

	Il semble également y avoir un problème avec l'OAEP\_padding sur le 
	chiffrement RSA. Bill Nickless recommande l'utilisation de PKCS\_padding.\\
	
	
	\textbf{Source :} \\
	\href{http://sourceforge.net/p/trousers/bugs/126/}
	{sourceforge.net/p/trousers/bugs/126/}
	
	
	
	
	
	
	
	
	
\section{Failles au niveau de l'authentification, la signature et la vérification}

	L'Université du Michigan a réussi l'exploit de récupérer la clé privée 
	d'un certificat RSA en un peu plus de 100h.\\
	L'attaque fonctionne par injection de fautes sur la méthode 
	d'authentification. La technique est donc très poussée, mais le résultat 
	en vaut le détour.\\

	Voici un petit tutoriel sur l'injection de faute : \\
	\href{http://rdist.root.org/2008/03/10/advances-in-rsa-fault-attacks/}
	{rdist.root.org/2008/03/10/advances-in-rsa-fault-attacks/}

	L'injection de faute doit se faire sur quelques bits pour ne pas
	disfonctionner le système tout entier. Les signatures erronées produites
	révèleront de l'information sur la clé privée. 
	Avec le bon matériel et 100h d'attente, la clé peut être reforgé.\\

	La cause venait de l'algorithme d'exponentiation modulaire (Fixed-Window 
	Exponentiation), qui a l'inconvénient d'utiliser plus de 1000 multiplications.
	La multiplication étant très sensible en cas de dégradation du microprocesseur.\\
	
	"The fixed-window exponentiation algorithm in the OpenSSL library does not 
	validate the correctness of the signature produced before sending it to the 
	client, a vulnerability that we exploit in our attack"\\

	\textbf{Sources} :\\
	\href{http://web.eecs.umich.edu/~valeria/research/publications/DATE10RSA.pdf}
	{web.eecs.umich.edu/~valeria/research/publications/DATE10RSA.pdf}\\
	\href{http://www.theregister.co.uk/2010/03/04/severe\_openssl\_vulnerability/}
	{www.theregister.co.uk/2010/03/04/severe\_openssl\_vulnerability/}\\
	
	Malheureusement pour pouvoir exploiter cette faille il faut pouvoir
	contrôler la machine (en ayant un accès au BIOS par exemple).
	
	En 2008, une vulnérabilité sur la malformation des signatures survient sur 
	OpenSSL (re-analysé en Novembre 2012) :\\

	\textbf{Sources} : \\
	\href{https://www.openssl.org/news/secadv\_20090107.txt}
	{www.openssl.org/news/secadv\_20090107.txt}\\
	\href{http://web.nvd.nist.gov/view/vuln/detail?vulnId=CVE-2008-5077}
	{web.nvd.nist.gov/view/vuln/detail?vulnId=CVE-2008-5077}\\


	Dans les recommandations générales, il est clairement indiqué que ce sont 
	les clients qui ne doivent plus utiliser une ancienne version d'OpenSSL ou 
	alors ne pas utiliser de certificats DSA/ECDSA\\

	Reste à savoir si c'est toujours d'actualité (e.g. si la faille est toujours
	exploitable), et si les serveurs respectent bien la recommandation.

	En 2009, même cas trouvé dans un autre protocole (NTP) avec la même 
	fonction EVP\_VerifyFinal :\\
	\href{http://www.cvedetails.com/cve/CVE-2009-0021/}
	{www.cvedetails.com/cve/CVE-2009-0021/}\\

	\textbf{Autres sources} :\\
	\href{http://cwe.mitre.org/data/definitions/599.html}
	{cwe.mitre.org/data/definitions/599.html}

	\newpage
	
	
	
	
	
	
	
	
\section{NSA, NIST et RSA - une back-door dans nos systèmes cryptographiques?}.

	\subsection{L'affaire Snowden et les documents top secrets de la NSA}
	
	Coup d'éclat en 2012, Edward Snowden, ancien-membre de la NSA et de
	la CIA,	dévoile l'existence des backdoors ainsi qu'un lot
	d'informations conséquent sur la forte affluence de la NSA
	sur le NIST et la RSA.\\

	\textbf{Source} : \\
	\href{http://www.reuters.com/article/2013/09/23/us-usa-security-snowden-rsa-idUSBRE98M06Q20130923}
	{www.reuters.com/article/2013/09/23/us-usa-security-snowden-rsa-idUSBRE98M06Q20130923}\\
	
	Il préleva ainsi plus de 1.700.000 documents de la NSA (d'après un 
	officier de la NSA - 15 décembre 2013), dont 31.000 
	ultra-confidentiels.\\
	
	Il délivra quelques documents à plusieurs journaux populaires
	tels que "The Guardian" et "The New York Times".\\
	
	Parmis les documents top secrets rendus publiques, un en particulier
	nous intéresse, il concerne le contrôle de la NSA sur les systèmes
	de chiffrement actuels, nom de code BULLRUN,
	voici quelques points importants :
	\begin{itemize}
		\item Insert vulnerabilities into commercial encryption systems, IT 
		systems, networks, endpoint communications devices used by targets
		\item Influence policies, standards and specification for commercial 
		public key technologies\\
	\end{itemize}

	\textbf{Sources concernant le projet BULLRUN} :
	\href{http://s3.documentcloud.org/documents/784159/sigintenabling-clean-1.pdf}
	{s3.documentcloud.org/documents/784159/sigintenabling-clean-1.pdf}	
	\href{http://fr.wikipedia.org/wiki/R\%C3\%A9v\%C3\%A9lations\_d\%27Edward\_Snowden}
	{fr.wikipedia.org/wiki/R\%C3\%A9v\%C3\%A9lations\_d\%27Edward\_Snowden}\\
	
	Bruce Schneier, un des plus grands cryptologues actuel,
	et fervent détracteur de la NSA, publie plusieurs articles concernant
	ce contrôle d'informations sur la question "La NSA a t-elle 
	réellement placé une backdoor au sein d'un nouveau système de 
	chiffrement?" \\

	\textbf{Sources} : \\
	\href{http://www.wired.com/politics/security/commentary/securitymatters/2007/11/securitymatters\_1115}\\
	{www.wired.com/politics/security/commentary/securitymatters/2007/11/securitymatters\_1115}\\
	
	Il évoque également le projet BULLRUN, montre comment la NSA
	peut placer ses backdoors, comment elle les choisient, et propose 
	plusieurs stratégies de défense pour les vaincre : \\
	\begin{itemize}
	\item Les vendeurs doivent rendre au minimum le code du chiffrement
	publique (spécifications concernant les protocoles inclus). Le reste
	peut être conservé secret. Afin d'en détecter les vulnérabilités.
	\item La communauté des cryptologues doit pouvoir offrir une version
	compatible et indépendante du système de chiffrement, en open-source
	ou en vente auprès des entreprises privées (pour financer les
	universités par exemple).
	\item Aucun secret! Tout doit être entièrement transparent auprès des
	clients.
	\item L'ensemble des PRNG doivent être rendus conformes avant 
	publication et acceptation.
	\item Aucune fuite d'informations n'est permise, surtout au niveau
	des protocoles de chiffrement. Ceci afin d'éviter la prédiction de
	clés privées.
	\end{itemize}
	
	\textbf{Source} :\\
	\href{https://www.schneier.com/blog/archives/2013/10/defending_again_1.html}
	{/www.schneier.com/blog/archives/2013/10/defending\_again\_1.html}\\
	
	
	En Septembre 2013, Matthew Green publie un article sur ce vaste
	problème entre la NSA et la sécurité cryptographique, qui
	a été salué par plusieurs cryptologues dont B. Schneier.\\
	
	Il précise cependant que ça ne reste que des spéculations, mais
	qu'elles sont nécessaire afin de doubler d'effort dans la sécurité
	de nos communications.\\

	\textbf{Source} :
	\href{http://blog.cryptographyengineering.com/2013/09/on-nsa.html}
	{blog.cryptographyengineering.com/2013/09/on-nsa.html}
	
	\subsection{La NSA verse 10M\$ à la RSA Company}

	Un rapport de Snowden, indique que la NSA a déversé plus de 10.000.000\$ à 
	la compagnie RSA pour qu'elle utilise ce dernier, et discutable, 
	algorithme comme générateur. \\
	On comprend donc mieux les suscipsions autour d'un accord entre le NIST et 
	la NSA pour la publication d'une recommandation de cet algorithme\\

	\textbf{Source} : \\
	\href{http://www.techienews.co.uk/973955/report-nsa-paid-rsa-10m-use-dual-ec-drbg-preferred-random-number-generator/}
	{www.techienews.co.uk/973955/report-nsa-paid-rsa-10m-use-dual-ec-drbg-preferred-random-number-generator/}\\

	\subsection{Le NIST et l'algo Dual EC DRBG}
	
	Les recommandations du NIST en matière de PRNG (qu'ils appelent plutôt 
	DRNG - Determinist Random Number Generation), débute en 2006, la 
	publication du dernier document sur les DRNG date de Janvier 2012 avec la 
	SP800-90A :\\
	\href{http://csrc.nist.gov/publications/nistpubs/800-90A/SP800-90A.pdf}
	{csrc.nist.gov/publications/nistpubs/800-90A/SP800-90A.pdf}\\

	Ce document présente quatres algorithmes de PRNG qui sont : 
	\begin{itemize}
		\item Le Hash\_DRBG basé sur des fonctions de hachage
		\item Le HMAC\_DRBG basé également sur des fonctions de hachage
		\item Le CTR\_DRBG basé sur du chiffrement par bloc
		\item Le Dual Elliptic Curve Deterministic RBG (ou Dual EC DRBG) basé 
		sur une théorie mathématique\\
	\end{itemize}

	Les trois premiers sont conventionnels, acceptés par toute la communauté 
	des cryptologues, et s'avère efficace car ils générent "suffisamment" 
	d'entropie.\\
	
	Le dernier est très différent des trois autres, dans le sens où il utilise 
	une fonction de chiffrement à sens unique. Certains cryptologues ont 
	démontrés que cet algorithme posséde des failles (d'autres indiquent 
	clairement que c'est une back-door du NIST...).\\
	
	En effet, on peut accepter l'utilisation d'une fonction à sens unique, à 
	condition que le secret utilisé ne soit pas conservé ailleurs (en d'autres 
	termes qu'il soit détruit).\\

	Que faire si le NIST garde le secret des algorithmes permettant 
	d'affaiblir considérablement le Dual EC DRGB, et rendre l'aléatoire 
	prévisible pour qui s'en donne les moyens?\\

	En recoupant plusieurs sources, le doute augmente considérablement.\\

	En 2006, Berry Schoenmakers et Andrey Sidorenko établissent une 
	cryptanalyse du DUAL\_EC\_DRGB.\\

	\textbf{Source} : \\
	\href{http://www.propublica.org/documents/item/786216-cryptanalysis-of-the-	dual-elliptic-curve}
	{www.propublica.org/documents/item/786216-cryptanalysis-of-the-dual-elliptic-curve}\\	
	
	
	En 2007, Dan Shumow et Niels Ferguson furent les premiers à dénoncer le 
	NIST d'avoir placer une backdoor délibérement dans cet algorithme.\\

	\textbf{Source} : \\
	\href{http://rump2007.cr.yp.to/15-shumow.pdf}
	{rump2007.cr.yp.to/15-shumow.pdf}
	
	Avant Septembre 2013, tout cela n'était que suspicion, mais depuis le 
	NIST à publié un bulletion de nouvelles recommandations pour les DRNG, et 
	indique (surtout grâce à un forcing de la communauté cryptologue) que le Dual EC 
	DRBG ne doit plus être utilisé pour les raisons suivantes :
	\begin{itemize}
		\item La provenance des points par défaut de la courbe elliptique 
		utilisée n'est pas clairement détaillée
		\item La génération de ces courbes n'est pas digne de confiance\\
	\end{itemize}

	D'où découle la recommandation suivante : \\
	\textit{"\textbf{Recommending against the use of SP 800-90A Dual Elliptic 
	Curve Deterministic Random Bit Generation:} NIST strongly recommends that, 
	pending the resolution of the security concerns and there - issuance of SP 
	800-90A, the Dual\_EC\_DRBG, as specified in the January 2012 version of 
	SP 800-90A, no longer be used"}\\

	\textbf{Source} : \\
	\href{http://csrc.nist.gov/publications/nistbul/itlbul2013\_09\_supplemental.pdf}
	{csrc.nist.gov/publications/nistbul/itlbul2013\_09\_supplemental.pdf}\\	
	
	\subsection{... Et OpenSSL ?}
	
	
	Maintenant, il faut rechercher l'utilisation de cet algorithme dans les 
	classes d'OpenSSL. La nouvelle recommandation du NIST faisant foi.\\

	Le directeur technique d'OpenSSL : Steve Marques a posté le 19 décembre 
	2013 : "Un bug inusuel a été détecté sur une situation inusuelle". \\
	
	L'implantation du DUAL\_EC\_DRGB dans OpenSSL contient une faille, celle-
	ci causant un arrêt brutal ou un blocage de programme.\\
	Le bug a toujours été là, et il vient seulement d'être détecté.\\
	Heureusement, personne n'a pu utiliser cet algorithme (à vérifier tout de 
	même!), celui-ci est resté dans les phases de tests, les passant tout de 
	même avec succès.\\

	\textbf{Source} : \\
	\href{https://lwn.net/Articles/578375/}{lwn.net/Articles/578375/}\\

	Source principal : \\
	\textbf{Source} : \\
	\href{http://nakedsecurity.sophos.com/2013/12/22/the-openssl-software-bug-that-saves-you-from-surveillance/}
	{nakedsecurity.sophos.com/2013/12/22/the-openssl-software-bug-that-saves-you-from-surveillance/}

	\begin{tabbing}
		\hspace{10cm}\=\kill
		\textbf{Utilité} : 8/10 \>	\textbf{Date} : Décembre 2013\\
	\end{tabbing}

	\newpage
	
	
	
	
\section{Failles sur les protocoles SSL et TLS}

\subsection{SSLv2}


\subsection{SSLv3 et TLSv1.0}

	En Septembre 2011, une attaque en man in the middle très efficace
	a vu le jour contre les protocoles SSLv3 et TLSv1.0.\\
	
	L'attaque est à clair choisi. Le but étant d'insérer des morceaux de
	texte clair grâce au navigateur dans la requête chiffré avec ces
	protocoles, ceci afin de récupérer les cookies de session.\\
	
	La technique est basique, un individu enregistre plusieurs cookies de
	session auprès de divers sites officiels (banques, messageries,
	etc...). Puis, il clique malencontreusement sur du code Java 
	malveillant (publicité, image, etc...). Et là, l'attaque se déroule
	automatiquement, l'ensemble des cookies est envoyé au serveur 
	malveillant qui n'a plus qu'à déchiffrer les clés de session.\\
	
	La cause viendrait du mode de chiffrement choisi : CBC.\\

	SSL/TLS est un protocole qui chiffre un canal de communication.\\
	De ce fait il ne chiffre pas un fichier unique, mais une série
	d'enregistrements.\\
	Il y a deux façon d'utiliser le mode CBC dans ce cas précis.
	\begin{itemize}
	\item Prendre chacun de ces enregistrements indépendamment des autres.
	Générer un nouveau vecteur d'initialisation à chaque fois.
	\item Traiter ces enregistrements comme un seul objet en les
	concaténant. Le vecteur d'initialisation est donc choisi aléatoirement
	pour le premier enregistrement et pour les autres, il aura pour
	valeur le dernier bloc de l'enregistrement précédent.
	\end{itemize}
	
	SSLv3 et TLS 1.0 utilisent ce deuxième choix, cela soulève 
	un lourd problème de sécurité.\\
	
	En 2004, Moeller trouve une méthode pour exploiter ce mauvais choix
	afin de récupérer des morceaux de textes clairs.\\
	
	
	\textbf{Sources} : \\
	\href{http://www.educatedguesswork.org/2011/09/security_impact_of_the_rizzodu.html}
	{www.educatedguesswork.org/2011/09/security\_impact\_of\_the\_rizzodu.html}	\\
	\href{https://www.imperialviolet.org/2011/09/23/chromeandbeast.html}
	{www.imperialviolet.org/2011/09/23/chromeandbeast.html}\\
	\href{http://www.theregister.co.uk/2011/09/19/beast_exploits_paypal_ssl/?page=2}
	{www.theregister.co.uk/2011/09/19/beast\_exploits\_paypal\_ssl/?page=2}
	\href{http://arstechnica.com/business/2011/09/new-javascript-hacking-tool-can-intercept-paypal-other-secure-sessions/}\\
	{arstechnica.com/business/2011/09/new-javascript-hacking-tool-can-intercept-paypal-other-secure-sessions/}\\
	
	Un étudiant de l'Université de Versailles à développer un logiciel
	nommé BEAST en javascript (les sources sont introuvables).\\
	
	La vidéo de l'attaque est accessible sur Youtube au lien ci dessous :\\
	\textbf{Source} : \\
	
	\href{http://www.youtube.com/watch?v=ujz4SXzWK9o}
	{www.youtube.com/watch?v=ujz4SXzWK9o}
	
	Alors certes il y a une faille immense, mais peu exploitable.\\
	Les grandes entreprises sont au courant (normalement) qu'il ne faut
	pas utiliser le mode CBC pour du chiffrement SSL/TLS.\\
	Et, dans tout les cas, plusieurs navigateurs ne permettent pas ce
	type d'attaque (c'est le cas de Chrome par exemple).\\
	
	
	
	\subsection{Validation des certificats SSL sans navigateur}	
	
	Six chercheurs des universités de Stanford et d'Austin au Texas, 
	analyse une attaque en Man in the Middle autour des certificats
	SSL sans utilisation d'un navigateur.\\
	
	Le titre est sans appel "Le code le plus dangereux du monde".\\
	
	\textbf{Source} :
	\href{http://www.cs.utexas.edu/~shmat/shmat_ccs12.pdf}
	{www.cs.utexas.edu/~shmat/shmat\_ccs12.pdf}\\
	
	SSL doit permettre d'être sécurisé en toutes circonstances, que le
	cache DNS soit empoisonné, que les attaquants contrôlent les
	points d'accès et les routeurs, etc...\\
	
	Il assure théoriquement trois grands principes de la
	cryptologie: la confidentialité, l'intégrité et
	\textbf{l'authentification}.

	Nous connaissons certaines failles au niveau du navigateur et de
	l'implantation SSL (voir ci-dessus).\\
	
	Mais il existe également d'autres cas d'utilisation du protocole SSL.
	Par exemple:
	\begin{itemize}
	\item Administration à distance basé sur le cloud, stockage 
	sécurisé sur le cloud en local.
	\item Transmissions de données sensibles (ex: e-commerce)
	\item Services en lignes comme les messageries électroniques
	\item Authentifications via applications mobiles comme Android et iOS
	\end{itemize}
	
	L'étude montre que la validation des certificats SSL est cassée
	sur plusieurs applications et librairies dont :
	\begin{itemize}
	\item OpenSSL
	\item JSSE
	\item CryptoAPI
	\item NSS
	\item GnuTLS
	\item etc...\\
	\end{itemize} 
	En fait un attaquant en Man In The Middle peut intercepter le
	secret entre un client et un serveur utilisant une connexion SSL.
	Il peut ainsi, récupérer des numéros de carte bancaire, avoir accès
	à une messagerie, récupérer des mots de passes, etc...\\
	
	La cause principale vient du fait que les développeurs retouchent
	les librairies cryptographiques à leurs façons. En voulant réparer
	un bug ou en souhaitant rendre SSL compatible avec leurs API, ils
	injectent de nouvelles vulnérabilités.\\
	Le pire est que l'application est souvent propriétaire et payante.\\
	
	Que ce soit accidentel ou intentionnel, l'une des conséquences 
	les plus grave et la non-validation de certificat
	sur des contexte où la sécurité est primordiale (e.g. 
	payement en ligne).\\
	
	La faute ne revient pas directement au code d'OpenSSL, mais à une
	mauvaise utilisation des différentes fonctions et options.\\
	
	Voici quelques exemples concrets concernant différentes API : 
	\begin{itemize}
	\item Les services comme Amazon's Flexible Payments Service PHP 
	et PayPal Payments Standard PHP passe le paramètre
	CURLOPT_SSL_VERIFYHOST à true alors que la valeur doit être passé à 2.
	La conséquence est la désactivation de la validation du certificat
	
	\item Lynx un navigateur textuel très connu et souvent utiliser dans
	le développement d'applications vérifie les certificats auto-signés
	seulement si la fonction de validation de certificat GnuTLS retourne
	une valeur négative. Malheureusement, dans certains cas la fonction
	peut retourner 0 pour certaines erreurs (dont les certificats signés
	par une autorité sans confiance).
	
	\item La librairie SSLSocketFactory de JSSE, très réputée, ne fait pas
	de vérification si la cypher suite du client vaut NULL ou est une 
	chaîne vide.
	
	\item Vulnérabilités sur Apache HttpClient, WebSockets, Android, ...
	
	\item Autres causes célèbres : non reconnaissance des expressions
	régulières, non vérification du résultat de la validation, 
	désactivation de l'authentification.\\
	\end{itemize}
	
	Les chercheurs nous donnent alors plusieurs leçons à retenir.
	Dont voici quelques points concernant la sécurité pouvant être
	apportée\\
	\begin{itemize}
	\item Premièrement, les vulnérabilités doivent être trouvées
	et réparées lors des phases de tests. Certaines se trouvent 
	très facilement si les procédures de tests sont bien réalisées.
	\item Deuxièmement, la plupart des librairies SSL ne sont pas
	\textbf{sûres par défaut}, laissant le choix de la sécurité aux 
	applications de plus haut niveau avec choix des options,
	choix de la vérification de l'hôte, choix d'interprétation
	des résultats.
	\item Troisièmement, même les librairies SSL sûrs par défaut
	peuvent être mal utilisées par des développeurs changeant les
	paramètres par défaut par des paramètres non sécurisés. La 
	cause peut venir d'une \textbf{mauvaise documentation} 
	ou d'une mauvaise formalisation de la part de l'API. 
	Les API devraient entre autres proposer des abstractions de 
	haut niveau pour les développeurs comme des tunnels
	d'authentification, plutôt que de les laisser traiter 
	des détails de bas niveau comme la vérification du nom d'hôte.\\
	\end{itemize}
	
	OpenSSL ne déroge pas à la règle.\\
	Voici quelques vulnérabilités du code :
	\begin{itemize}
	\item Les contraintes de nom x509 ne sont pas correctement validés.
	\item Les applications DOIVENT fournir elles même leurs code de
	vérification de nom d'hôte. Or, des protocoles comme HTTPS, LDAP
	ont chacun leurs propres notions de validations.
	\item Un programme utilisant OpenSSL peut exécuter la fonction
	SSL\_connect pour le handshake SSL. Bien que certaines erreurs de
	validation soient signalés par SSL\_connect, d'autres ne peuvent
	être vérifier qu'en appelant la fonction SSL\_get\_verify\_result,
	alors que SSL\_connect se contente de retourner "OK".
	\end{itemize}
	
	Exemple de mauvaise utilisation : Trillian\\
	Trillian est une messagerie cliente instantanée	relié à OpenSSL
	pour la sécurisation de l'établissement de connexion.
	Par défaut OpenSSL ne soulève pas d'exception en cas de certificat
	auto-signé ou de non-confiance auprès de la chaîne de vérification.
	A la place, il envoi un drapeau. De plus, il ne vérifie jamais le
	nom d'hôte.\\
	
	Si l'application appel la fonction SSL_CTX_set pour initialiser
	le drapeau SSL_VERIFY_PEER, alors SSL_connect se ferme et affiche un
	message d'erreur lorsque le certificat n'est pas valide.\\
	Mais Trillian n'initialise jamais ce drapeau.\\
	Par conséquence, SSL_connect va retourner 1 et le statut de la
	validation du certificat peut être connu en appelant la fonction
	SSL_get_verify_result.\\
	Encore une fois, Trillian n'appel pas cette fonction.\\
	
\section{ANNEXE A - État de l'art : Génération de l'aléa}
	
	\textbf{NIST - Toolkit} : \\
	\href{http://csrc.nist.gov/groups/ST/toolkit/rng/index.html}
	{csrc.nist.gov/groups/ST/toolkit/rng/index.html}\\
	\href{http://csrc.nist.gov/groups/ST/toolkit/random\_number.html}
	{csrc.nist.gov/groups/ST/toolkit/random\_number.html}\\

	\begin{itemize}
		\item Téléchargement de logiciels (PRNG)
		\item Documentation
		\item Batterie de tests
		\item Analyse statistique
		\item Description de RNG
	\end{itemize}
	
	\begin{tabbing}
		\hspace{10cm}\=\kill
		\textbf{Utilité} : 8/10 \>	\textbf{Date} : Décembre 2013\\
	\end{tabbing}

	\textbf{NIST - Publications} :\\		
	\href{http://csrc.nist.gov/publications/PubsFIPS.html}
	{csrc.nist.gov/publications/PubsFIPS.html}\\
	\href{http://csrc.nist.gov/publications/fips/fips140-2/fips1402.pdf}
	{csrc.nist.gov/publications/fips/fips140-2/fips1402.pdf}\\
	\href{http://csrc.nist.gov/publications/drafts/800-90/draft-sp800-90c.pdf}
	{csrc.nist.gov/publications/drafts/800-90/draft-sp800-90c.pdf}\\

	\begin{itemize}
		\item Publications et Drafts du NIST sur sécurité informatique 
		(dont FIPS)
		\item FIPS 140-1 140-2 140-3
	\end{itemize}
	
	\begin{tabbing}
		\hspace{10cm}\=\kill
		\textbf{Utilité} : 8/10 \>	\textbf{Date} : Août 2013\\
	\end{tabbing}


	\textbf{INRIA} : \\
	\href{http://hal.archives-ouvertes.fr/docs/00/73/86/38/PDF/rr8060.pdf}
	{hal.archives-ouvertes.fr/docs/00/73/86/38/PDF/rr8060.pdf}\\

	\begin{itemize}
		\item Transfert d'entropie au sein du RNG de Linux
		\item Collecte d'entropie. Comment?
		\item Tests\\
	\end{itemize}

	\textbf{Utilité} : 9/10\\

	\textbf{Publication de l'université de Jérusalem} : \\
	\href{http://eprint.iacr.org/2006/086.pdf}
	{eprint.iacr.org/2006/086.pdf}\\

	\begin{itemize}
		\item Analyse du RNG de Linux
		\item Collecte et analyse d'entropie
		\item Recommandations\\
	\end{itemize}
	
	\begin{tabbing}
		\hspace{10cm}\=\kill
		\textbf{Utilité} : 9/10 \>	\textbf{Date} : 2006\\
	\end{tabbing}

	\textbf{INTEL LABS} : \\
	\href{https://crypto.stanford.edu/RealWorldCrypto/slides/jesse.pdf}
	{crypto.stanford.edu/RealWorldCrypto/slides/jesse.pdf}\\

	\begin{itemize}
		\item Graine des RNG
		\item Problème similaire à notre 1e partie
		\item Procédure de génération des clés RSA mise en doute
		\item Liste de recommandations
		\item Source d'entropie
	\end{itemize}
		
	\begin{tabbing}
		\hspace{10cm}\=\kill
		\textbf{Utilité} : 9/10 \>	\textbf{Date} : 2013-2014\\
	\end{tabbing}

	\textbf{Source} : \\
	\href{http://www.linuxfromscratch.org/hints/downloads/files/entropy.txt}
	{www.linuxfromscratch.org/hints/downloads/files/entropy.txt}\\

	\begin{itemize}
		\item /dev/random VS /dev/urandom
		\item Entropie et sécurité cryptographique
		\item Obtenir plus d'entropie avec Glibc et RNG-tools
		\item Modification de l'entropie dans OpenSSL
		\item Tester la qualité de l'entropie\\
	\end{itemize}

	\begin{tabbing}
		\hspace{10cm}\=\kill
		\textbf{Utilité} : 9/10 \>	\textbf{Date} : Mai 2007\\
	\end{tabbing}

	\textbf{TOR} : \\
	\href{https://lists.torproject.org/pipermail/tor-talk/2013-December/031483.html}
	{lists.torproject.org/pipermail/tor-talk/2013-December/031483.html}\\

	\begin{itemize}
		\item Bug sur les RNG OpenSSL (entre autres)
	\end{itemize}

	\begin{tabbing}
		\hspace{10cm}\=\kill
		\textbf{Utilité} : 9/10 \>	\textbf{Date} : Décembre 2013\\
	\end{tabbing}

	\textbf{Calomel} : \\
	\href{https://calomel.org/entropy\_random\_number\_generators.html}
	{calomel.org/entropy\_random\_number\_generators.html}\\

	\begin{itemize}
		\item Bon tutoriel sur l'entropie et la génération de nombre 
		aléatoire (PRNG ou DRBG)
		\item PRNG's state, exemple : /dev/random
		\item Quel problème avec les PRNG? Analyse de random, urandom, 
		arandom, srandom
		\item Il propose sa solution de PRNG
		\item Il teste les PRNG avec ENT
		\item Deux qualités d'un PRNG = vitesse + qualité entropie
		\item MAIS surtout qualité d'entropie! La crypto est trop sensible!
		\item Augmenter l'entropie sur Linux avec RNGD (Julien a trouvé 
		mieux) On passe de 150 o/s $\leftarrow$ 4096 o/s (Julien 100 Mb/
		s)
		\item NIST SP800-90, FIPS 140-2, and ANSI X9.82
	\end{itemize}
	
	\begin{tabbing}
		\hspace{10cm}\=\kill
		\textbf{Utilité} : 8/10 \>	\textbf{Date} : Août 2013\\
	\end{tabbing}

	\textbf{Henric} : \\
	\href{http://www.henric.info/random/}{www.henric.info/random/} \\

	\begin{itemize}
		\item Bon récapitulatif
		\item Liens vers NIST
		\item Tests statistiques (NIST, DIEHARD, TESTU01, ...)
		\item c7random
		\item Ajout d'entropie avec VIA PadLock OpenSSL Patch
	\end{itemize}
		
	\begin{tabbing}
		\hspace{10cm}\=\kill
		\textbf{Utilité} : 7/10 \>	\textbf{Date} : 2007-2008\\
	\end{tabbing}

	\textbf{International Computer Institute} :

	\href{http://ube.ege.edu.tr/~mutaf/random.pdf}
	{ube.ege.edu.tr/~mutaf/random.pdf}\\

	\begin{itemize}
		\item Généralités
		\item Statistiques d'aléatoire, entropie, prédiction
		\item Problèmes avec certains PRNG
		\item PRNG supposés forts
		\item Standards\\
	\end{itemize}

	\textbf{Utilité} : 5/10\\

	\textbf{The POSSE Project} : \\
	\href{http://www.cis.upenn.edu/~dsl/POSSE/}
	{www.cis.upenn.edu/~dsl/POSSE/}\\

	\begin{itemize}
		\item Equipe de chercheurs en sécurité
		\item Audit de plusieurs logiciels open source dont OpenSSL
		\item Articles
	\end{itemize}
	
	\begin{tabbing}
		\hspace{10cm}\=\kill
		\textbf{Utilité} : 5/10 \>	\textbf{Date} : 2003\\
	\end{tabbing}

	\textbf{Blog - Analyse de l'API Random d'OpenSSL} : \\
	\href{http://jbp.io/2014/01/16/openssl-rand-api/}
	{jbp.io/2014/01/16/openssl-rand-api/}\\

	\begin{itemize}
		\item Rand\_bytes() et Rand\_pseudo\_bytes()
		\item Tests de non-sûreté
		\item Recommandations et patchs de sécurité
	\end{itemize}
	
	\begin{tabbing}
		\hspace{10cm}\=\kill
		\textbf{Utilité} : 9/10 \>	\textbf{Date} : Janvier 2014\\
	\end{tabbing}
	
	\textbf{Blog - La cryptographie d'OpenSSL est-elle cassée?} : \\
	\href{http://www.linuxadvocates.com/2013/09/is-openssls-cryptography-broken.html}
	{www.linuxadvocates.com/2013/09/is-openssls-cryptography-broken.html}\\

	\begin{itemize}
		\item NSA can break internet Encryption (RSA ?)
		\item Courbes elliptiques, et choix d'aléa
		\item Constantes générées par un employée de la NSA
		\item A prendre avec des pincettes (trouver des sources fiables)
	\end{itemize}
		
	\begin{tabbing}
		\hspace{10cm}\=\kill
		\textbf{Utilité} : 8/10 \>	\textbf{Date} : Septembre 2013\\
	\end{tabbing}

		\textbf{Tutoriel - Utiliser l'API OpenSSL Random Number} : \\
		\href{http://etutorials.org/Programming/secure+programming/Chapter+11.+Random+Numbers/11.9+Using+the+OpenSSL+Random+Number+API/}
		{etutorials.org/Programming/secure+programming/Chapter+11.+Random+Numbers/11.9+Using+the+OpenSSL+Random+Number+API/}\\

		\begin{itemize}
			\item Problème
			\item Solution
			\item Discussion
			\item Listing des fonctions de génération d'aléatoires
		\end{itemize}

	\begin{tabbing}
		\hspace{10cm}\=\kill
		\textbf{Utilité} : 7/10 \>	\textbf{Date} : 2013\\
	\end{tabbing}
	
	\subsection{Forums - Discussions intéressantes autour de la randomisation}

		\textbf{Source : } \href{http://marc.info/?l=openssl-dev\&m=132733475012928\&w=2}
		{marc.info/?l=openssl-dev\&m=132733475012928\&w=2}
		
		\begin{tabbing}
			\hspace{10cm}\=\kill
			\textbf{Utilité} : 6/10 \>	\textbf{Date} : Janvier 2012\\
		\end{tabbing}

		\textbf{Source : } \href{http://openssl.6102.n7.nabble.com/understanding-openssl-entropy-td42000.html}
		{openssl.6102.n7.nabble.com/understanding-openssl-entropy-td42000.html}
		
		\begin{tabbing}
			\hspace{10cm}\=\kill
			\textbf{Utilité} : 8/10 \>	\textbf{Date} : Février 2012\\
		\end{tabbing}

		\textbf{Source} : \href{http://openssl.6102.n7.nabble.com/seed-RANDFILE-confusion-td19793.html}
		{openssl.6102.n7.nabble.com/seed-RANDFILE-confusion-td19793.html}
	
		\begin{tabbing}
			\hspace{10cm}\=\kill
			\textbf{Utilité} : 7/10 \>	\textbf{Date} : Octobre 2012\\
		\end{tabbing}

		\textbf{Source} : \href{http://crypto.stackexchange.com/questions/9412/what-to-watch-for-with-openssl-generating-weak-keys-low-entropy}
		{crypto.stackexchange.com/questions/9412/what-to-watch-for-with-openssl-generating-weak-keys-low-entropy}
		
		\begin{tabbing}
			\hspace{10cm}\=\kill
			\textbf{Utilité} : 7/10 \>	\textbf{Date} : Juillet 2013\\
		\end{tabbing}

		\textbf{Source} : \href{http://security.stackexchange.com/questions/3259/howto-seed-the-	prng-in-openssl-properly}
		{security.stackexchange.com/questions/3259/howto-seed-the-prng-in-openssl-properly}\\

		\begin{tabbing}
			\hspace{10cm}\=\kill
			\textbf{Utilité} : 7/10 \>	\textbf{Date} : Avril 2011\\
		\end{tabbing}

		\textbf{Source} : \href{http://stackoverflow.com/questions/18349321/random-number-generator-and-seed}
		 {stackoverflow.com/questions/18349321/random-number-generator-and-seed}
		 
		\begin{tabbing}
			\hspace{10cm}\=\kill
			\textbf{Utilité} : 6/10 \>	\textbf{Date} : Septembre 2013\\
		\end{tabbing}

		\textbf{Source} : \href{http://security.stackexchange.com/questions/3259/howto-seed-the-prng-in-openssl-properly}
		{security.stackexchange.com/questions/3259/howto-seed-the-prng-in-openssl-properly}
		
		\begin{tabbing}
			\hspace{10cm}\=\kill
			\textbf{Utilité} : 7/10 \>	\textbf{Date} : Juillet 2013\\
		\end{tabbing}

	\subsection{Autres sources}

		[WIKI] CryptGenRandom de Windows : \href{http://en.wikipedia.org/wiki/CryptGenRandom}
		{en.wikipedia.org/wiki/CryptGenRandom}\\

		[WIKI] OpenSSL/Random Numbers : \href{http://en.wikibooks.org/wiki/OpenSSL/Random\_numbers}
		{en.wikibooks.org/wiki/OpenSSL/Random\_numbers}\\

		[BLOG] Bug OpenSSL/Debian : Accident ou Backdoor? \href{https://freedom-to-tinker.com/blog/kroll/software-transparency-debian-openssl-bug/}
		{freedom-to-tinker.com/blog/kroll/software-transparency-debian-openssl-bug/}\\

		[AUDIT] Vulnérabilité toujours d'actualité - mais non exploitable :		\href{http://www.digital-chaos.net/2014/01/openssl-story-about-some-old-source.html}
		{www.digital-chaos.net/2014/01/openssl-story-about-some-old-source.html}\\

		[FLAWS] OpenSSL Security Advisory - 2012 : \href{http://lwn.net/Articles/475009/}{lwn.net/Articles/475009/}\\

		[CWE] - Failles d'application (ex N-599, N-699) : \href{http://cwe.mitre.org/data/definitions/599.html}
		{cwe.mitre.org/data/definitions/599.html}

		\newpage
	

\setcounter{tocdepth}{2}

\end{document}