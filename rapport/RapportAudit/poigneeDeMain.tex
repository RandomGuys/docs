\chapter{Poignée de main}

\section{Définitions et contexte}

\section{Audits}
	\subsection{Audit 1 : X }
	\subsubsection{Norme visée}
	\subsubsection{Faille}
	\subsubsection{Implémentation}
		
		\paragraph{Version OpenSSL.\\}
		
		\paragraph{Fonction.\\}
		La fonction liée à cette norme est accessible sous le paquetage \texttt{bla/bla/bla}, dont les composantes principales sont listées en \textit{listing} \ref{codeAleatoire}.
		
		
		\begin{lstlisting}[style=customc,caption=codeAleatoire.c, label=codeAleatoire]
#include <stdio.h>
#define N 10
/* Block
 * comment */
 
int main()
{
    int i;
 
    // Line comment.
    puts("Hello world!");
 
    for (i = 0; i < N; i++)
    {
        puts("LaTeX is also great for programmers!");
    }
 
    return 0;
}
		\end{lstlisting}
		
		
		
		\paragraph{Audit.\\}




\section{Recommandations générales}




%Si nécessaire, Code de figure : 
\begin{figure}[H]
	\centering
	\includegraphics[scale=0.2]{images/logo_univ.png}
	\caption{Titre de figure 3.1}
	\label{fi31}
\end{figure}