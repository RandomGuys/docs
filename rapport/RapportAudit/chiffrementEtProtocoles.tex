\chapter{Chiffrement et Protocoles}
\section{Définitions et contexte}

\section{Audits}
	\subsection{Audit 1 : Les "Manger's attack" sur RSA-OAEP}
		\subsubsection{Normes visées}
	
		\subsubsection{Faille}
		
			RSA-OAEP peut être soumis à une attaque nommée "Mangers Attack" 
			selon son implantation \cite{mangers2010falko}. 
			OpenSSL semble être vulnérable à une attaque de ce type, à
			base de "prédictions".\\
			La vulnérabilité semble être très récente puisqu'elle fonctionne sous
			OpenSSL 1.0.0.\\
		
			La Technische Universität Darmstadt (Allemagne) apporte des contre-
			mesures, et note qu'il existe toutefois des cas où l'algorithme est 
			plus résistant.\\
			
			Il semble également y avoir un problème avec l'OAEP\_padding sur le 
			chiffrement RSA. Bill Nickless recommande l'utilisation de 
			PKCS\_padding. \cite{sourceforgeRSAbroken}	\\
		
		
		\subsubsection{Implémentation}
			
			\paragraph{Version OpenSSL.\\}
			
			\paragraph{Fonction.\\}
			La fonction liée à cette norme est accessible sous le paquetage
			\texttt{bla/bla/bla}, dont les composantes principales sont 
			listées en \textit{listing} \ref{codeAleatoire}.
		
		
		\begin{lstlisting}[style=customc,caption=codeAleatoire.c, label=codeAleatoire]
#include <stdio.h>
#define N 10
/* Block
 * comment */
 
int main()
{
    int i;
 
    // Line comment.
    puts("Hello world!");
 
    for (i = 0; i < N; i++)
    {
        puts("LaTeX is also great for programmers!");
    }
 
    return 0;
}
		\end{lstlisting}
		
	\subsection{Audit 2 : Chiffrement SSLv3 ou TLS 1.0 en mode CBC}
		\subsubsection{Normes visées}
	
	
		\subsubsection{Faille}
			
			\subsubsubsection{Description}
			
			En Septembre 2011, une attaque en man in the middle très efficace
			a vu le jour contre les protocoles SSLv3 et TLS 1.0.
			\cite{ekr2011beast} \cite{imperial2011beast} \cite{goodin2011beast}
			\cite{gallagher2011beast}\\
			
			L'attaque est à clair choisi. Le but étant d'insérer des morceaux de
			texte clair grâce au navigateur dans la requête chiffré avec ces
			protocoles, ceci afin de récupérer les cookies de session.\\
			
			La technique est basique, un individu enregistre plusieurs cookies de
			session auprès de divers sites officiels (banques, messageries,
			etc...). Puis, il clique malencontreusement sur du code Java 
			malveillant (publicité, image, etc...). Et là, l'attaque se déroule
			automatiquement, l'ensemble des cookies est envoyé au serveur 
			malveillant qui n'a plus qu'à déchiffrer les clés de session.\\
			
			La cause viendrait du mode de chiffrement choisi : CBC.\\
		
			SSL/TLS est un protocole qui chiffre un canal de communication.\\
			De ce fait il ne chiffre pas un fichier unique, mais une série
			d'enregistrements.\\
			Il y a deux façon d'utiliser le mode CBC dans ce cas précis.
			\begin{itemize}
			\item Prendre chacun de ces enregistrements indépendamment 
			des autres.
			Générer un nouveau vecteur d'initialisation à chaque fois.
			\item Traiter ces enregistrements comme un seul objet en les
			concaténant. Le vecteur d'initialisation est donc choisi
			aléatoirement pour le premier enregistrement et pour les autres, 
			il aura pour valeur le dernier bloc de l'enregistrement précédent.
			\end{itemize}
			
			SSLv3 et TLS 1.0 utilisent ce deuxième choix, cela soulève 
			un lourd problème de sécurité.\\
			
			En 2004, Moeller \cite{moeller2004cbc} trouve 
			une méthode pour exploiter ce mauvais choix
			afin de récupérer des morceaux de textes clairs.\\
			
			Alors certes il y a une faille immense, mais peu exploitable.\\
			Les grandes entreprises sont au courant (normalement) qu'il ne faut
			pas utiliser le mode CBC pour du chiffrement SSL/TLS.\\
			Et, dans tout les cas, plusieurs navigateurs ne permettent pas ce
			type d'attaque (c'est le cas de Chrome par exemple).\\
			
			\subsubsubsection{Tests}
			
			Nous n'avons pas repris les tests du logiciel BEAST qui s'avère
			être introuvable sur le Web (celui-ci étant un projet universitaire,
			développé par un étudiant de l'Université de Versailles).
			Mais une vidéo de l'exploit est accessible sur YouTube au lien
			ci-dessous : \\
			
			\textbf{Lien YouTube : } 
			\href{http://www.youtube.com/watch?v=ujz4SXzWK9o}
			{http://www.youtube.com/watch?v=ujz4SXzWK9o}
		
		\subsubsection{Implémentation}
			
			\paragraph{Version OpenSSL.\\}
			
			\paragraph{Fonction.\\}
			La fonction liée à cette norme est accessible sous 
			le paquetage \texttt{bla/bla/bla}, dont les composantes 
			principales sont listées en \textit{listing} \ref{codeAleatoire}.
		
		
		\begin{lstlisting}[style=customc,caption=codeAleatoire.c, label=codeAleatoire]
#include <stdio.h>
#define N 10
/* Block
 * comment */
 
int main()
{
    int i;
 
    // Line comment.
    puts("Hello world!");
 
    for (i = 0; i < N; i++)
    {
        puts("LaTeX is also great for programmers!");
    }
 
    return 0;
}
		\end{lstlisting}
		
	\subsection{Audit 3 : Non-validation des certificats SSL}
		\subsubsection{Normes visées}
	
		\subsubsection{Faille}
		
			Six chercheurs des universités de Stanford et d'Austin au Texas, 
			analyse une attaque en Man in the Middle autour des certificats
			SSL sans utilisation d'un navigateur.\\
			
			Le titre est sans appel "Le code le plus dangereux du monde".\\
			
			\textbf{Source} :
			\href{http://www.cs.utexas.edu/~shmat/shmat_ccs12.pdf}
			{www.cs.utexas.edu/~shmat/shmat\_ccs12.pdf}\\
			
			SSL doit permettre d'être sécurisé en toutes circonstances, que le
			cache DNS soit empoisonné, que les attaquants contrôlent les
			points d'accès et les routeurs, etc...\\
			
			Il assure théoriquement trois grands principes de la
			cryptologie: la confidentialité, l'intégrité et
			\textbf{l'authentification}.
		
			Nous connaissons certaines failles au niveau du navigateur et de
			l'implantation SSL (voir ci-dessus).\\
			
			Mais il existe également d'autres cas d'utilisation du protocole SSL.
			Par exemple:
			\begin{itemize}
			\item Administration à distance basé sur le cloud, stockage 
			sécurisé sur le cloud en local.
			\item Transmissions de données sensibles (ex: e-commerce)
			\item Services en lignes comme les messageries électroniques
			\item Authentifications via applications mobiles comme Android et iOS
			\end{itemize}
			
			L'étude montre que la validation des certificats SSL est cassée
			sur plusieurs applications et librairies dont :
			\begin{itemize}
			\item OpenSSL
			\item JSSE
			\item CryptoAPI
			\item NSS
			\item GnuTLS
			\item etc...\\
			\end{itemize} 
			En fait un attaquant en Man In The Middle peut intercepter le
			secret entre un client et un serveur utilisant une connexion SSL.
			Il peut ainsi, récupérer des numéros de carte bancaire, avoir accès
			à une messagerie, récupérer des mots de passes, etc...\\
			
			La cause principale vient du fait que les développeurs retouchent
			les librairies cryptographiques à leurs façons. En voulant réparer
			un bug ou en souhaitant rendre SSL compatible avec leurs API, ils
			injectent de nouvelles vulnérabilités.\\
			Le pire est que l'application est souvent propriétaire et payante.\\
			
			Que ce soit accidentel ou intentionnel, l'une des conséquences 
			les plus grave et la non-validation de certificat
			sur des contexte où la sécurité est primordiale (e.g. 
			payement en ligne).\\
			
			La faute ne revient pas directement au code d'OpenSSL, mais à une
			mauvaise utilisation des différentes fonctions et options.\\
			
			Voici quelques exemples concrets concernant différentes API : 
			\begin{itemize}
			\item Les services comme Amazon's Flexible Payments Service PHP 
			et PayPal Payments Standard PHP passe le paramètre
			CURLOPT\_SSL\_VERIFYHOST à true alors que la valeur doit être 
			passé à 2.
			La conséquence est la désactivation de la validation du certificat
			
			\item Lynx un navigateur textuel très connu et souvent utiliser dans
			le développement d'applications vérifie les certificats auto-signés
			seulement si la fonction de validation de certificat GnuTLS retourne
			une valeur négative. Malheureusement, dans certains cas la fonction
			peut retourner 0 pour certaines erreurs (dont les certificats signés
			par une autorité sans confiance).
			
			\item La librairie SSLSocketFactory de JSSE, très réputée, ne fait pas
			de vérification si la cypher suite du client vaut NULL ou est une 
			chaîne vide.
			
			\item Vulnérabilités sur Apache HttpClient, WebSockets, Android, ...
			
			\item Autres causes célèbres : non reconnaissance des expressions
			régulières, non vérification du résultat de la validation, 
			désactivation de l'authentification.\\
			\end{itemize}
			
			Les chercheurs nous donnent alors plusieurs leçons à retenir.
			Dont voici quelques points concernant la sécurité pouvant être
			apportée\\
			\begin{itemize}
			\item Premièrement, les vulnérabilités doivent être trouvées
			et réparées lors des phases de tests. Certaines se trouvent 
			très facilement si les procédures de tests sont bien réalisées.
			\item Deuxièmement, la plupart des librairies SSL ne sont pas
			\textbf{sûres par défaut}, laissant le choix de la sécurité aux 
			applications de plus haut niveau avec choix des options,
			choix de la vérification de l'hôte, choix d'interprétation
			des résultats.
			\item Troisièmement, même les librairies SSL sûrs par défaut
			peuvent être mal utilisées par des développeurs changeant les
			paramètres par défaut par des paramètres non sécurisés. La 
			cause peut venir d'une \textbf{mauvaise documentation} 
			ou d'une mauvaise formalisation de la part de l'API. 
			Les API devraient entre autres proposer des abstractions de 
			haut niveau pour les développeurs comme des tunnels
			d'authentification, plutôt que de les laisser traiter 
			des détails de bas niveau comme la vérification du nom d'hôte.\\
			\end{itemize}
			
			OpenSSL ne déroge pas à la règle.\\
			Voici quelques vulnérabilités du code :
			\begin{itemize}
			\item Les contraintes de nom x509 ne sont pas correctement validés.
			\item Les applications DOIVENT fournir elles même leurs code de
			vérification de nom d'hôte. Or, des protocoles comme HTTPS, LDAP
			ont chacun leurs propres notions de validations.
			\item Un programme utilisant OpenSSL peut exécuter la fonction
			SSL\_connect pour le handshake SSL. Bien que certaines erreurs de
			validation soient signalés par SSL\_connect, d'autres ne peuvent
			être vérifier qu'en appelant la fonction SSL\_get\_verify\_result,
			alors que SSL\_connect se contente de retourner "OK".
			\end{itemize}
			
			Exemple de mauvaise utilisation : Trillian\\
			Trillian est une messagerie cliente instantanée	relié à OpenSSL
			pour la sécurisation de l'établissement de connexion.
			Par défaut OpenSSL ne soulève pas d'exception en cas de certificat
			auto-signé ou de non-confiance auprès de la chaîne de vérification.
			A la place, il envoi un drapeau. De plus, il ne vérifie jamais le
			nom d'hôte.\\
			
			Si l'application appel la fonction SSL\_CTX\_set pour initialiser
			le drapeau SSL\_VERIFY\_PEER, alors SSL\_connect se ferme et affiche un
			message d'erreur lorsque le certificat n'est pas valide.\\
			Mais Trillian n'initialise jamais ce drapeau.\\
			Par conséquence, SSL\_connect va retourner 1 et le statut de la
			validation du certificat peut être connu en appelant la fonction
			SSL\_get\_verify\_result.\\
			Encore une fois, Trillian n'appel pas cette fonction.\\
			
		
		\subsubsection{Implémentation}
			
			\paragraph{Version OpenSSL.\\}
			
			\paragraph{Fonction.\\}
			La fonction liée à cette norme est accessible sous le paquetage
			\texttt{bla/bla/bla}, dont les composantes principales sont 
			listées en \textit{listing} \ref{codeAleatoire}.
		
		
		\begin{lstlisting}[style=customc,caption=codeAleatoire.c, label=codeAleatoire]
#include <stdio.h>
#define N 10
/* Block
 * comment */
 
int main()
{
    int i;
 
    // Line comment.
    puts("Hello world!");
 
    for (i = 0; i < N; i++)
    {
        puts("LaTeX is also great for programmers!");
    }
 
    return 0;
}
		\end{lstlisting}
		


\section{Recommandations générales}




%Si nécessaire, Code de figure : 
\begin{figure}[H]
	\centering
	\includegraphics[scale=0.2]{images/logo_univ.png}
	\caption{Titre de figure 2.1}
	\label{fig21}
\end{figure}