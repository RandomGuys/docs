\addcontentsline{toc}{section}{Conclusion}
\section*{Conclusion}

Cette étude fut l'occasion d'effectuer une première approche d'audit en analysant de façon gnénérale des failles liées à un des programmes majeurs dans le domaine de la cryptographie, OpenSSL. Nous avons pu déterminer une stratégie globale d'audit suivant les exigences et suggestions du client, puis nous avons effectué un approfondissement dans plusieurs domaines, répartis de façon modulaire entre les membres du projet.\\


Nous avons pu notamment constater que les failles identifiées sont principalement sur des anciennes versions d'OpenSSL et qu'elles sont, par conséquent, pour la plupart aujourd'hui corrigées. Toutefois, un utilisateur du logiciel possédant une version non à jour est vulnérable -suivant sa version- à des attaques référencées dans ce rapport. \\


L'audit de deux semaines nous a permis d'établir un état général sur les faiblesses de ce logiciel. Nous aurions aimé avoir davantage de temps pour entrer au plus près du code. Comme nous l'avons rappelé plusieurs fois, le projet OpenSSL n'a pas un code facilement lisible, et le manque de documentation nous a considérablement freiné dans nos analyses. De plus, le code en lui-même est de taille considérable, avec plus de 400 000 lignes de code.\\


De ce fait, nous considérons que le projet OpenSSL mériterait un rafraîchissement complet, en clarifiant à la fois la structure et l'implémentation, mais aussi en ajoutant une documentation digne de ce nom, afin d'en faciliter la compréhension et d'éviter au maximum toute mauvaise utilisation.  















