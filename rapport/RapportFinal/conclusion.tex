\addcontentsline{toc}{section}{Conclusion}
\section*{Conclusion}

\paragraph{Rétrospective :\\}
Nous pouvons retenir plusieurs points importants sur ce projet.\\
Premièrement, malgré les nombreuses études démontrant les vulnérabilités des certificats circulant sur Internet, elles persistent toujours et il devient urgent de bien sécuriser sa machine contre des attaques aussi efficaces et rapides (quelques jours à peine) face à des serveurs mal sécurisés.\\

Deuxièmement, nous avons en l'espace de quelques jours trouvé plusieurs vulnérabilités sur le code d'OpenSSL, et nous pouvons en conclure qu'il est nécessaire d'adopter rapidement une politique d'action plutôt qu'une politique de réaction.\\

Troisièmement, nous avons fait face à un code très mal documenté, avec un paramétrage par défaut contestable, et aucune classe de haut niveau. Il est au minimum important de retravailler la documentation.\\

Enfin, sur deux semaines nous n'avons pu faire qu'un audit de surface. Un audit complet, en profondeur, prendrait beaucoup plus de temps mais serait vraiment un sujet de projet intéressant.\\

\paragraph{Apports du projet :\\}
Le projet nous a apportés plus d'expériences dans plusieurs domaines (développement applicatif, scripts, web, base de données, réseaux, cryptographie, ...). Nous avons cherché à rendre un projet soigné, dont chaque partie serait liée et serait directement accessible sur une même plateforme.\\
Notre étude sur la deuxième partie nous a permis d'analyser en détail des articles scientifiques; il s'agissait d'un bon exercice de style pour notre futur professionnel. De plus, ce projet a attisé notre curiosité, et sachant qu'il ne faut pas se fier aux systèmes de sécurité mis en place, cela nous a donné l'envie à tous de continuer nos recherches sur les problématiques liés à la sécurité.

\paragraph{Axes d'améliorations :\\}
