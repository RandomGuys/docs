\chapter{Analyse dynamique}

\section{Objectifs}
L'objectif de ce projet consiste à étudier le comportement d'une machine cliente lorsqu'elle se connecte sur un serveur HTTPS. Bien qu'à la base considérée comme optionnelle, nous avons décidé de développer cette partie du projet car nous avions de l'avance sur les parties précédentes et que l'analyse du code d'OpenSSL nous a donné des idées sur l'implémentation de cet outil. \\

Nous avons développé un serveur HTTPS permettant d'analyser les différentes \textit{ciphersuite} utilisées à l'établissement de connexion d'un client à un serveur, et d'établir un diagnostique sur la qualité des suites utilisées par le client. Pour rappel, une \textit{ciphersuite} est une combinaison de noms pour l'échange des clefs, l'authentification, le chiffrement, et les algorithmes MAC utilisés pour négocier les paramètres de sécurité pour une connexion réseau par TLS (\textit{Transport Layer Security})/ SSL (\textit{Secure Sockets Layer}). La structure de ces suites est définie dans la RFC 5246 \cite{rfc5246} et fut également décrite dans le rapport d'audit d'OpenSSL (Chapitre 5, Protocoles SSL/TLS).\\



La figure \ref{systServer} illustre le système à mettre en place.
\begin{figure}[H]
\begin{center}
\begin{tikzpicture}[remember picture]
\begin{umlseqdiag} 
\umlobject[class=Navigateur] {C};
\umlobject[class=Serveur HTTPS,x=7] {S};

\begin{umlcall}[dt=6,op={TCP connect},return={TCP accept},padding=4]{C}{S}\end{umlcall}

\begin{umlcall}[dt=6,op={SSL connect},return={SSL accept},padding=4]{C}{S}\end{umlcall}

\begin{umlcall}[dt=6,op={GET /site/analyze HTTP/1.0},return={code HTML}]{C}{S}

\begin{umlcallself}[op={Récupérer version du protocole}]{S}\end{umlcallself} 
\begin{umlcallself}[op={Récupérer support de la compression TLS}]{S}\end{umlcallself} 
\begin{umlcallself}[op={Récupérer ciphersuite cliente}]{S}\end{umlcallself} 
\begin{umlcallself}[op={Récupérer algorithmes de signature clients}]{S}\end{umlcallself} 
\begin{umlcallself}[op={Récupérer courbes elliptiques clientes}]{S}\end{umlcallself} 
\begin{umlcallself}[op={Récupérer support ticket TLS}]{S}\end{umlcallself} 

\end{umlcall}

\end{umlseqdiag} 
\end{tikzpicture}
\end{center}
\caption{Schéma de l'analyse dynamique du navigateur client}
	\label{systServer}
\end{figure}



\section{Mise en place du serveur}
\subsection{Exigences}
Il convient de pouvoir identifier les caractéristiques de la connexion entre le client et le serveur à savoir :
\begin{itemize}
\item la version du protocole mise en œuvre ;
\item les \textit{ciphersuites} proposées par le client ;
\item les courbes elliptiques supportées ;
\item les algorithmes de signature ;
\item la compression TLS, qui permet de compresser les données chiffrées afin d'accélérer le transfert de celles-ci ;
\item l'activation ou non du ticket de session.\\
\end{itemize}

Suivant ces caractéristiques, nous pouvons émettre un jugement sur la qualité du client SSL/TLS utilisé. Si un client possède des attributs étant considérés comme vulnérables, alors il faut pouvoir les identifier rapidement sur le site, et expliciter les raisons de l'affaiblissement du système.  


\subsection{Faiblesses à identifier}
\subsubsection{Version du protocole}
Si le client utilise une version \texttt{TLS 1.0} ou \texttt{SSLv3}, alors celui-ci sera considéré comme non sécurisé.\\

\begin{enumerate}
\item La version \texttt{SSLv3} précède la version \texttt{TLS 1.0} ; elle est considérée comme non sûre de nos jours. 

 
\item La version \texttt{TLS 1.0} est la toute première version de TLS, et demande une pré-configuration client et serveur importante pour être utilisé de façon sécurisée. De plus, cette version ne permet pas d'utiliser les dernières \textit{ciphersuites} récentes offrant une meilleure sécurité. De plus, cette version est vulnérable à l'attaque BEAST	(\textit{Browser Exploit Against SSL/TLS}).

\item Enfin, la version \texttt{TLS 1.1}  n'étant pas la dernière version TLS, nous considérons qu'elle est moyennement sûre.\\
\end{enumerate}



\subsubsection{Ciphersuites}
Nous avons pu identifier toutes les \textit{ciphersuites} étant considérées comme peu sûres, à savoir :
\begin{itemize}
\item les \textit{ciphersuites} utilisant des clefs de taille inférieure à 128 bits pour le chiffrement ;
\item les \textit{ciphersuites} ne spécifiant aucun chiffrement pour la connexion ; 
\item les \textit{ciphersuites} sans authentification serveur, rendant l'attaque d'homme par le milieu possible ;
\item autres \textit{ciphersuites} étant supposées arrêtées après SSL 3.0, ou dont les caractéristiques de sûreté ne sont pas connues.
\end{itemize}


\subsubsection{Courbes elliptiques}
La RFC 5430 \ref{rfc5430} décrit deux niveaux de sécurité pour les courbes elliptiques, les clefs de 128 bits et le clesf de 192 bits, référencés sur le tableau \ref{ciphsuiteSec}. 

\begin{table}[H]
\centering

\begin{tabularx}{13cm}{|l|X|}
\hline
\textbf{\textit{Ciphersuite}} & \textbf{Niveau de sécurité} \\
\hline
\verb+TLS_ECDHE_ECDSA_WITH_AES_128_GCM_SHA256+ & 128\\
\verb+TLS_ECDHE_ECDSA_WITH_AES_128_CBC_SHA256+ & 128\\            
\verb+TLS_ECDHE_ECDSA_WITH_AES_128_CBC_SHA+    & 128\\            
\verb+TLS_ECDHE_ECDSA_WITH_AES_256_GCM_SHA384+ & 192\\            
\verb+TLS_ECDHE_ECDSA_WITH_AES_256_CBC_SHA384+ & 192\\            
\verb+TLS_ECDHE_ECDSA_WITH_AES_256_CBC_SHA+    & 192\\
\hline
\end{tabularx}

\caption{Niveaux de sécurité pour les \textit{ciphersuites}}
\label{ciphsuiteSec}
\end{table}


La RFC recommande l'utilisation de deux courbes :
\begin{itemize}
\item pour un niveau de sécurité à 128 bits, la courbe \texttt{secp256r1};
\item pour un niveau de sécurité à 192 bits, la courbe \texttt{secp384r1}.
\end{itemize}



\subsubsection{Compression des données chiffrées}
La compression des données chiffrées permet d'accélérer le transfert des données. Toutefois, cette technique permet à un attaquant d'effectuer une attaque de type CRIME (\textit{Compression Ratio Info-leak Made Easy}), qui permet de récupérer des données de session. %


\subsubsection{Ticket de session}
Les tickets de session permettent d'accélérer la reprise de communication chiffrée avec le serveur. Le temps de la poignée de main peut prendre du temps et ce ticket permet d'établir une session entre le client et le serveur. Lorsqu'un client se déconnecte puis se reconnecte avec un ticket de session, les échanges du \textit{handshake} sont passés, gagnant un temps considérable. 



\section{Implémentation}
Pour la partie TCP, nous avons utilisé une bibliothèque écrite en Licence 3 et simplifiée pour notre utilisation : SocketTCP.

Le serveur HTTPS a été développé en C et il implémente les méthodes suivantes :
\begin{itemize}
\item \verb+get_version (SSL *ssl)+, qui retourne la version de connexion utilisée ; 
\item \verb+get_ecc_list (SSL *ssl)+, qui retourne la liste des courbes elliptiques ; 
\item \verb+get_cipher_suite_string (SSL_CIPHER *c)+, qui retourne le texte associé à une \textit{ciphersuite} ; 
\item \verb+get_cipher_suite_list (SSL *ssl)+, qui permet l'affichage de toutes les \textit{ciphersuites} ; 
\item \verb+get_sig_algs (SSL *ssl)+, qui renvoie les algorithmes de signature ; 
\item \verb+get_analyze_page (SSL *ssl)+ qui génère la réponse complète (en-tête et corps HTTP);
\item \verb+handle_connection (void *param)+  qui gère la connexion entre le serveur et le client : gestion des ressources statiques (CSS, JS) et de l'appel à l'analyse dynamique (\verb+get_analyze_page+) ;
\item \verb+new_thread (SocketTCP *socket)+ qui gère le lancement des threads pour chaque connexion.\\
\end{itemize}

Pour développer un tel serveur, il a simplement fallu s'appuyer sur un serveur HTTP basique puis rajouter la surcouche SSL dans \verb+handle_connection+ avec mise en place d'un contexte SSL et lecture/écriture chiffrée.\\


Pour le rendu de la page, nous sommes restés consistant avec le site développé lors du sprint 2 et avons donc utilisé le bootstrapt twitter. De plus, il nous permet d'indiquer les niveaux de sécurité avec des badges colorés qui permettent d'identifier visuellement et rapidement les problèmes.
