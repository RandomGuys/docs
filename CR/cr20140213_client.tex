\documentclass[a4paper,11pt,french]{article}
\usepackage[utf8]{inputenc}

\usepackage[T1]{fontenc}
\usepackage[francais]{babel} 
\usepackage[top=1.5cm, bottom=2cm, left=1.5cm, right=1.5cm, includeheadfoot]{geometry} %pour les marges
\usepackage{lmodern}
\usepackage{tikz}	
%\usepackage{tikz-uml}
\usepackage{fancyhdr} % Required for custom headers
\usepackage{lastpage} % Required to determine the last page for the footer
\usepackage{extramarks} % Required for headers and footers
\usepackage{graphicx} % Required to insert images
\usepackage{tabularx, longtable}
\usepackage{color, colortbl}
\usepackage{amsmath}
\usepackage{amssymb}
\usepackage{mathrsfs}
\usepackage{amsfonts}
\usepackage{amsthm}
\usepackage{gensymb}
\usepackage{cmbright}
\usepackage[hidelinks]{hyperref}
\usepgflibrary{arrows} % for pgf-umlsd

\linespread{1.1} % Line spacing

% Set up the header and footer
\pagestyle{fancy}
\lhead{\textbf{\hmwkClass -- \hmwkSubject \\ \hmwkTitle \\ \hmwkDocName}} % Top left header
\rhead{\includegraphics[width=10em]{logo_univ.png}}
\lfoot{\lastxmark} % Bottom left footer
\cfoot{} % Bottom center footer
\rfoot{Page\ \thepage\ / \pageref{LastPage}} % Bottom right footer
\renewcommand\headrulewidth{0.4pt} % Size of the header rule
\renewcommand\footrulewidth{0.4pt} % Size of the footer rule

\setlength{\headheight}{40pt}

\newcommand{\hmwkTitle}{Audit des implantations SSL/TLS} % Assignment title
\newcommand{\hmwkClass}{Master 2 SSI } % Course/class
\newcommand{\hmwkAuthorName}{X} % Your name
\newcommand{\hmwkSubject}{Conduite de projet} % Subject
\newcommand{\hmwkDocName}{Compte-rendu de réunion client du 13/02/14} % Document name

\newcommand{\version}{X} % Document version
\newcommand{\docDate}{X} % Document date
\newcommand{\checked}{X} % Checker name
\newcommand{\approved}{X} % Approver name

\definecolor{gris}{rgb}{0.95, 0.95, 0.95}

\author{\hmwkAuthorName}
\date{} % Insert date here if you want it to appear below your name

\usepackage{array}
\usepackage{tabularx}

\begin{document}
\pagestyle{fancy}

\begin{center}\textbf{\Huge{\hmwkDocName}}\end{center}
	
\section{Présents}
\begin{itemize}
\item Ayoub Otmani (client)
\item Pascal Edouard 
\item Claire Smets
\item Julien Legras (Réd. CR) 
\item Mathieu Latimer
\item William Boisseleau
\end{itemize}

\section{Objet}
Ce document reprend les points évoqués durant la réunion du 13 février 2014 avec le client. Il s'agissait de faire un point sur ce qui a été audité depuis le début du sprint pour savoir quels points devaient être approfondis.

\section{Contenu abordé}
\begin{center}

\begin{tabular}{ | l | p{10cm} | p{2cm} | l |}
	\hline
   \textbf{Sujet} & \textbf{Actions à entreprendre} & \textbf{Assignation} & \textbf{Date limite}  \\ \hline
L'aléatoire & Ne pas implémenter des tests mais plutôt tester la génération de l'aléatoire dans openssl avec des outils comme diehard & Pascal, William & 20/02/2014\\ \hline
OAEP & Analyser le procédé de la Manger's attack & Mathieu, William & 20/02/2014\\ \hline
SSL & Tester s'il est possible de forcer l'utilisation d'un algorithme faible & Julien & 20/02/2014\\ \hline
Rapport d'audit& \begin{itemize}\item Décrire l'idée synthétique des attaques trouvées \item Conclure sur une politique de bonnes pratiques et donner des solutions et alternatives à openssl\end{itemize} & Équipe & 20/02/2014\\
 \hline
 \end{tabular}
 
\end{center}
\end{document}