\documentclass[a4paper,11pt,french]{article}
\usepackage[utf8]{inputenc}

\usepackage[T1]{fontenc}
\usepackage[francais]{babel} 
\usepackage[top=1.5cm, bottom=2cm, left=1.5cm, right=1.5cm, includeheadfoot]{geometry} %pour les marges
\usepackage{lmodern}
\usepackage{pict2e}
\usepackage{tikz}        
%\usepackage{tikz-uml}
\usepackage{fancyhdr} % Required for custom headers
\usepackage{lastpage} % Required to determine the last page for the footer
\usepackage{extramarks} % Required for headers and footers
\usepackage{graphicx} % Required to insert images
\usepackage{tabularx, longtable}
\usepackage{color, colortbl}
\usepackage{amsmath}
\usepackage{amssymb}
\usepackage{mathrsfs}
\usepackage{amsfonts}
\usepackage{amsthm}
\usepackage{gensymb}
%\usepackage{cmbright}
\usepgflibrary{arrows} % for pgf-umlsd

\linespread{1.1} % Line spacing

% Set up the header and footer
\pagestyle{fancy}
\lhead{\textbf{\hmwkClass -- \hmwkSubject \\ \hmwkTitle \\ \hmwkDocName}} % Top left header
\rhead{\includegraphics[width=10em]{logo_univ.png}}
\lfoot{\lastxmark} % Bottom left footer
\cfoot{} % Bottom center footer
\rfoot{Page\ \thepage\ / \pageref{LastPage}} % Bottom right footer
\renewcommand\headrulewidth{0.4pt} % Size of the header rule
\renewcommand\footrulewidth{0.4pt} % Size of the footer rule

\setlength{\headheight}{40pt}

\newcommand{\hmwkTitle}{Audit des implantations SSL/TLS} % Assignment title
\newcommand{\hmwkClass}{Master 2 SSI } % Course/class
\newcommand{\hmwkAuthorName}{X} % Your name
\newcommand{\hmwkSubject}{Conduite de projet} % Subject
\newcommand{\hmwkDocName}{Compte-rendu de réunion du 24/01/14} % Document name

\newcommand{\version}{X} % Document version
\newcommand{\docDate}{X} % Document date
\newcommand{\checked}{X} % Checker name
\newcommand{\approved}{X} % Approver name

\definecolor{gris}{rgb}{0.95, 0.95, 0.95}

\author{\hmwkAuthorName}
\date{} % Insert date here if you want it to appear below your name

\usepackage{array}
\usepackage{tabularx}

\begin{document}
\pagestyle{fancy}

\begin{center}\textbf{\Huge{\hmwkDocName}}\end{center}
        
\section{Présents}
\begin{itemize}
\item Pascal Edouard  
\item Claire Smets
\item Julien Legras
\item Mathieu Latimer (Réd. CR)
\item William Boisseleau
\end{itemize}

\section{Objet}
Compte-rendu d'Audit du 24 Janvier 2014 avec M. Abdellah Godard. Lors de cette réunion, nous avons étudié et relevé les dernières modifications à apporter aux 
documents livrables suivants : "Spécification technique du besoin", "Analyse des risques", "Plan de développement".
\newline
\newline
Nous avons également parlé de l'organisation des comptes-rendus, et un point a été soulevé : "À qui doit-on assigner les actions à entreprendre?".
\newline
\newline
La prochaine réunion avec M. Abdellah Godard se fera aux alentours du 21 Février 2014, et elle sera également le dernier Audit du projet, avant sa livraison finale.
Nous évoquerons alors les tests effectués (cf. document "Cahier de recettes"), et la validité du client sur les différents livrables .

\section{Contenu abordé}
\begin{center}

\begin{tabular}{ | p{3cm} | p{11cm} | p{3cm} |}
        \hline
   \textbf{Sujet} & \textbf{Actions à entreprendre} & \textbf{Réalisation}   \\ \hline
 STB & 1 - Définir le niveau des exigences (facultatif, recommandé, obligatoire) & Claire \\ 
 & 2 - Indiquer qu'on ne traite plus le cas de SSH dans le UseCase 1 (et non dans les exigences) &   \\ \hline
ADR & 1 - Revoir la syntaxe du document & Pascal\\
        & 2 - Redéfinir les plans d'actions (notamment : qui fait quoi) & \\
        & 3 - Découper la tâche absence en plusieurs sous-tâches & \\ \hline
PDD/Outils projet & 1 - Avoir une vue globale du projet à montrer au client (pour la soutenance finale) - notamment rendre visible les charges et les ressources sur chaque tâche & Pascal \\
		 & 2 - Mettre les documents Teambox à jour (chemin critique, dépendances, gestion du temps et des ressources, délais) & \\
		& 3 - Insérer des jalons de validation (retour client, remarques du client sur la finalité des livrables auprès de l'AMOA) & \\
		& 4 - La tâche Test doit être mise au pluriel (et éventuellement y ajouter optimisations) & \\
		& 5 - Tester l'outil Gantter Project & 
		\\ \hline
CR &  Nouvelle colonne indiquant les acteurs réalisant les actions à entreprendre & Les prochains rédacteurs de CR \\ \hline
Client &  Noter les prochaines modifications du client sur un document à part, le faire signer et le valider & Claire \\ \hline
Autres &  Rendre visible les différences entre les versions des documents (sur-lignage par exemple) & Rédacteurs respectifs de documents \\ \hline
 \end{tabular}
 
\end{center}
\end{document}
