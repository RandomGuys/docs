\documentclass[a4paper,11pt,french]{article}
\usepackage[utf8]{inputenc}

\usepackage[T1]{fontenc}
\usepackage[francais]{babel} 
\usepackage[top=1.5cm, bottom=2cm, left=1.5cm, right=1.5cm, includeheadfoot]{geometry} %pour les marges
\usepackage{lmodern}
\usepackage{pict2e}
\usepackage{tikz}	
%\usepackage{tikz-uml}
\usepackage{fancyhdr} % Required for custom headers
\usepackage{lastpage} % Required to determine the last page for the footer
\usepackage{extramarks} % Required for headers and footers
\usepackage{graphicx} % Required to insert images
\usepackage{tabularx, longtable}
\usepackage{color, colortbl}
\usepackage{amsmath}
\usepackage{amssymb}
\usepackage{mathrsfs}
\usepackage{amsfonts}
\usepackage{amsthm}
\usepackage{gensymb}
%\usepackage{cmbright}
\usepgflibrary{arrows} % for pgf-umlsd

\linespread{1.1} % Line spacing

% Set up the header and footer
\pagestyle{fancy}
\lhead{\textbf{\hmwkClass -- \hmwkSubject \\ \hmwkTitle \\ \hmwkDocName}} % Top left header
\rhead{\includegraphics[width=10em]{logo_univ.png}}
\lfoot{\lastxmark} % Bottom left footer
\cfoot{} % Bottom center footer
\rfoot{Page\ \thepage\ / \pageref{LastPage}} % Bottom right footer
\renewcommand\headrulewidth{0.4pt} % Size of the header rule
\renewcommand\footrulewidth{0.4pt} % Size of the footer rule

\setlength{\headheight}{40pt}

\newcommand{\hmwkTitle}{Audit des implantations SSL/TLS} % Assignment title
\newcommand{\hmwkClass}{Master 2 SSI } % Course/class
\newcommand{\hmwkAuthorName}{X} % Your name
\newcommand{\hmwkSubject}{Conduite de projet} % Subject
\newcommand{\hmwkDocName}{Compte-rendu de réunion du 20/12/13} % Document name

\newcommand{\version}{X} % Document version
\newcommand{\docDate}{X} % Document date
\newcommand{\checked}{X} % Checker name
\newcommand{\approved}{X} % Approver name

\definecolor{gris}{rgb}{0.95, 0.95, 0.95}

\author{\hmwkAuthorName}
\date{} % Insert date here if you want it to appear below your name

\usepackage{array}
\usepackage{tabularx}

\begin{document}
\pagestyle{fancy}

\begin{center}\textbf{\Huge{\hmwkDocName}}\end{center}
	
\section{Présents}
\begin{itemize}
\item Pascal Edouard (Chef de projet)
\item Claire Smets (Relation Client)
\item Julien Legras
\item Mathieu Latimer
\item William Boisseleau (Réd. CR)
\end{itemize}

\section{Objet}
Ce document reprend les points évoqués durant la réunion du 20 décembre 2013. Il s'agissait notamment de rependre et corriger certains éléments du projet rapportés dans notre documentation. 

\section{Contenu abordé}
\begin{center}

\begin{tabular}{ | l | p{15cm} |}
	\hline
   \textbf{Sujet} & \textbf{Actions à entreprendre}    \\ \hline
 STB & Fournir un diagramme UC général \\ 
 & \textbf{UC1 :} origine de la récupération des certificats à déterminer et indiquer/définir l'administrateur \\
 & \textbf{UC2 et UC5 :} corriger le user/client et définir si le serveur est considéré comme un acteur \\ 
 & \textbf{UC4 :} effectuer une analyse statique \\ 
 & \textbf{Exigences :} revoir les notions de priorité et supprimer "présentation claire et visible". Supprimer E1, car il n'y a pas d'exigence d'interface. Concernant EQ2, préciser en quoi l'application doit être optimisée. \\ \hline
CDR &  Confirmer les exigences pour la deuxième partie \\ \hline
DAL & Concernant l'architecture applicative, créer un schéma général et justifier les choix de configuration\\
	& Faire apparaitre les fonctions dans les dépendances de la description des constituants (supprimer userMachine)\\
	& Décrire les fonctions, algorithmes, détailler et argumenter\\ \hline
ADR	& \textbf{Criticité :} expliciter/argumenter son calcul \\ \hline
PDD & \textbf{Tâches :}  prendre des actions préventives plutôt qu'un rétroplanning.\\
 & \textbf{SCRUM :}  argumenter son utilisation\\
 & \textbf{Description des tâches :} mentionner le partionnement des items. \\
 & \textbf{Attribution des charges :} supprimer le TOUS, décrire les sous-tâches. Modifier le Gant et refaire le découpage des charge (itérations doivent être équivalentes). Détailler les outils utilisés. \\
 & \textbf{Gestion de la documentation :} préciser l'utilisation de GIT et teambox\\ \hline
 \end{tabular}
 
\end{center}

\end{document}