\documentclass[a4paper,11pt,french]{article}
\usepackage[utf8]{inputenc}

\usepackage[T1]{fontenc}
\usepackage[francais]{babel} 
\usepackage[top=1.5cm, bottom=2cm, left=1.5cm, right=1.5cm, includeheadfoot]{geometry} %pour les marges
\usepackage{lmodern}
\usepackage{pict2e}
\usepackage{tikz}	
\usepackage{fancyhdr} % Required for custom headers
\usepackage{lastpage} % Required to determine the last page for the footer
\usepackage{extramarks} % Required for headers and footers
\usepackage{graphicx} % Required to insert images
\usepackage{tabularx, longtable}
\usepackage{color, colortbl}

\usepackage{cmbright}
\usepgflibrary{arrows} % for pgf-umlsd

\linespread{1.1} % Line spacing

% Set up the header and footer
\pagestyle{fancy}
\lhead{\textbf{\hmwkClass -- \hmwkSubject \\ \hmwkTitle \\ \hmwkDocName}} % Top left header
\rhead{\includegraphics[width=10em]{logo_univ.png}}
\lfoot{\lastxmark} % Bottom left footer
\cfoot{} % Bottom center footer
\rfoot{Page\ \thepage\ / \pageref{LastPage}} % Bottom right footer
\renewcommand\headrulewidth{0.4pt} % Size of the header rule
\renewcommand\footrulewidth{0.4pt} % Size of the footer rule

\setlength{\headheight}{40pt}

\newcommand{\hmwkTitle}{Audit des implantations SSL/TLS} % Assignment title
\newcommand{\hmwkClass}{Master 2 SSI } % Course/class
\newcommand{\hmwkAuthorName}{Pascal Edouard} % Your name
\newcommand{\hmwkSubject}{Conduite de projet} % Subject
\newcommand{\hmwkDocName}{Analyse des risques} % Document name

\newcommand{\version}{1.0} % Document version
\newcommand{\docDate}{04/12/2013} % Document date
\newcommand{\checked}{Julien Legras} % Checker name
\newcommand{\approved}{Ayoub Otmani} % Approver name

\definecolor{gris}{rgb}{0.95, 0.95, 0.95}

\author{\hmwkAuthorName}
\date{} % Insert date here if you want it to appear below your name


\begin{document}
\pagestyle{fancy}

\vspace*{5cm}
\begin{center}\textbf{\Huge{\hmwkDocName}}\end{center}
\vspace*{7cm}
	
\begin{center}
\fcolorbox{black}{gris}{
\begin{minipage}{10cm}
\begin{tabularx}{10cm}{lXl}
	\bfseries{Version} & & \version\\
	& & \\
	\bfseries{Date} & & \docDate\\
	& & \\
	\bfseries{Rédigé par} & & \hmwkAuthorName \\
	& & \\
	\bfseries{Relu par} & & \checked \\
	& & \\
	\bfseries{Approuvé par} & & \approved \\
	& & \\
\end{tabularx}
\end{minipage}
}
\end{center}

\newpage

%Tableau de mises à jour
\vspace*{1cm}
\begin{center}
\textbf{\huge{MISES À JOUR}}\\
\vspace*{3cm}
	\begin{tabularx}{16cm}{|c|c|X|}
	\hline
	\bfseries{Version} & \bfseries{Date} & \bfseries{Modifications réalisées}\\
	\hline
	1.0 & 4/12/2013 & Création\\
	\hline
	\end{tabularx}
\end{center}

%La table des matières
\clearpage
\tableofcontents
\clearpage
\section{Objet}
Document réunissant les différents risques qui pourraient arriver pendant ce projet ainsi que des plans d'actions pour les risques MAJEUR/CRITIQUE.

\section{Terminologie et sigles utilisés}
\begin{itemize}
\item Calcul de la criticité :\\
{\centering CRITICITÉ = PROBABILITÉ $\times$ IMPACT}
\item Table des correspondances PROBABILITÉ/valeur et IMPACT/valeur :

\begin{center}
\begin{tabular}{|l|l|c|l|l|}
\hline
\textbf{Nom probabilité}&\textbf{Valeur}&&\textbf{Nom impact}&\textbf{Valeur}\\
\hline
FAIBLE&2&&MINEUR&4\\
\hline
MAJEUR&3&&MAJEUR&5\\
\hline
FORT&4&&CRITIQUE&6\\
\hline
\end{tabular}
\end{center}
\item []
\item Env: environnement
\item RH : ressources humaines
\end{itemize}
\section{Registre des risques}

\begin{center}
\begin{tabular}{|c|p{5cm}|p{3cm}|c|c|c|c|}
\hline
\textbf{Réf.} & \textbf{Description} & \textbf{Facteurs} & \textbf{Type} & \textbf{Probabilité} & \textbf{Impact} & \textbf{Criticité} \\
\hline
 R01 & Absence prolongée (plusieurs semaines) d'un ou plusieurs membre de l'équipe. &
 Accidents, Maladie, voyage à l'étranger imprévu. &
 Humain &
 Majeur &
 Majeur &
 15 \\
 \hline
 R02 & Accumulation du retard dans le travaille de William. &
 Emploi du temps trop chargé avec les cours à l'INSA et les tâches du project. &
 Humain &
 Majeur &
 Mineur &
 12 \\
 \hline
 R03 & Programme de récupération des certificats ne fonctionne pas. &
 Mauvaise implémentation, bug intempestif, aucune connexion internet.  &
 Technique &
 Faible &
 Majeur &
 10 \\
\hline
 R04 & Programme de factorisation ne fonctionne pas. &
 Mauvaise implémentation, bug intempestif, ressources (mémoire, processeur, etc) insuffisantes.  &
 Technique &
 Fort &
 Majeur &
 15 \\
\hline
 R05 & Perte des données. &
 Panne serveur GIT, récupération impossible des données depuis le serveur. &
 Technique &
 Faible &
 Majeur &
 10 \\
\hline

\end{tabular}
\end{center}

\section{Plans d'actions}
\begin{center}
\begin{tabular}{|l|p{8cm}|l|l|}
\hline
\textbf{Réf.}&\textbf{Action(s) prévue(s)}&\textbf{Action effectuée}&\textbf{Date}\\
\hline
	R01 & 
	Ré-organisation et répartition des tâches entre les membres de l'équipe présents. & & \\
\hline
	R02 &
	Repartir ses tâches en retard à un ou plusieurs membres de l'équipe pour ratrapper le retard. & & \\
\hline
	R03 &
	Re-travailler la struture du programme, répartir les bugs existant aux membres de l'équipe, changer de réseau. & & \\
\hline
R04 &
	Re-travailler la structure du programme, répartir les bugs existant aux membres de l'équipe, discuter d'une ressource disponible pour exécuter le programme de calcule.& & \\
\hline
R05 &
	Faire une évaluation des données restants sur les machines de chaque membres de l'équipe et calculer les pertes de données. & & \\
\hline
%R06 & & & \\
%\hline
\end{tabular}
\end{center}
\end{document}