\documentclass[a4paper,11pt,french]{article}
\usepackage[utf8]{inputenc}

\usepackage[T1]{fontenc}
\usepackage[francais]{babel}
\usepackage[top=1.5cm, bottom=2cm, left=1.5cm, right=1.5cm, includeheadfoot]{geometry} %pour les marges
\usepackage{lmodern}
\usepackage{tikz}        
\usepackage{fancyhdr} % Required for custom headers
\usepackage{lastpage} % Required to determine the last page for the footer
\usepackage{extramarks} % Required for headers and footers
\usepackage{graphicx} % Required to insert images
\usepackage{tabularx, longtable}
\usepackage{color, colortbl}

\usepackage{cmbright}
\usepgflibrary{arrows} % for pgf-umlsd

\linespread{1.1} % Line spacing

% Set up the header and footer
\pagestyle{fancy}
\lhead{\textbf{\hmwkClass -- \hmwkSubject \\ \hmwkTitle \\ \hmwkDocName}} % Top left header
\rhead{\includegraphics[width=10em]{logo_univ.png}}
\lfoot{\lastxmark} % Bottom left footer
\cfoot{} % Bottom center footer
\rfoot{Page\ \thepage\ / \pageref{LastPage}} % Bottom right footer
\renewcommand\headrulewidth{0.4pt} % Size of the header rule
\renewcommand\footrulewidth{0.4pt} % Size of the footer rule

\setlength{\headheight}{40pt}

\newcommand{\hmwkTitle}{Audit des implantations SSL/TLS} % Assignment title
\newcommand{\hmwkClass}{Master 2 SSI } % Course/class
\newcommand{\hmwkAuthorName}{Pascal Edouard} % Your name
\newcommand{\hmwkSubject}{Conduite de projet} % Subject
\newcommand{\hmwkDocName}{Analyse des risques} % Document name

\newcommand{\version}{1.3} % Document version
\newcommand{\docDate}{28/01/2014} % Document date
\newcommand{\checked}{Julien Legras, Mathieu Latimier} % Checker name
\newcommand{\approved}{Ayoub Otmani} % Approver name

\definecolor{gris}{rgb}{0.95, 0.95, 0.95}

\author{\hmwkAuthorName}
\date{} % Insert date here if you want it to appear below your name


\begin{document}
\pagestyle{fancy}

\vspace*{5cm}
\begin{center}\textbf{\Huge{\hmwkDocName}}\end{center}
\vspace*{7cm}
        
\begin{center}
\fcolorbox{black}{gris}{
\begin{minipage}{10cm}
\begin{tabularx}{10cm}{lXl}
        \bfseries{Version} & & \version\\
        & & \\
        \bfseries{Date} & & \docDate\\
        & & \\
        \bfseries{Rédigé par} & & \hmwkAuthorName \\
        & & \\
        \bfseries{Relu par} & & \checked \\
        & & \\
        \bfseries{Approuvé par} & & \approved \\
        & & \\
\end{tabularx}
\end{minipage}
}
\end{center}

\newpage

%Tableau de mises à jour
\vspace*{1cm}
\begin{center}
\textbf{\huge{MISES À JOUR}}\\
\vspace*{3cm}
        \begin{tabularx}{16cm}{|c|c|X|}
        \hline
        \bfseries{Version} & \bfseries{Date} & \bfseries{Modifications réalisées}\\
        \hline
        1.0 & 4/12/2013 & Création\\
        \hline
        1.1 & 22/01/2014 & Risque SSH\\
        \hline
        2.0 & 23/01/2014 & Description de criticité\\
        \hline
        2.1 & 28/01/2014 & Redéfinition des plans d'actions, et découpage des tâches d'absences\\
        \hline
        2.2 & 7/02/2014 & Résolution bugs sur programme de factorisation\\
        \hline
        \end{tabularx}
\end{center}

%La table des matières
\clearpage
\tableofcontents
\clearpage
\section{Objet}
Document réunissant les différents risques qui pourraient arriver pendant ce projet ainsi que des plans d'actions pour les risques MAJEUR/CRITIQUE.

\section{Terminologie et sigles utilisés}
\begin{itemize}
\item Calcul de la criticité :\\

Afin de déterminer la criticité de chaque risque, une évaluation de la probabilité d'occurrence et de l'impact est estimé. La criticité est le produit de la probabilité de son occurrence par l'impact que le risque a sur le projet. On sera donc prêt pour appliquer les mesures adéquates face à ces risques. \\

{\centering CRITICITÉ = PROBABILITÉ $\times$ IMPACT} \\

\item Table des correspondances PROBABILITÉ/valeur et IMPACT/valeur :\\

\begin{center}
\begin{tabular}{|l|l|c|l|l|}
\hline
\textbf{Nom probabilité}&\textbf{Valeur}&&\textbf{Nom impact}&\textbf{Valeur}\\
\hline
FAIBLE&2&&MINEUR&4\\
\hline
MAJEUR&3&&MAJEUR&5\\
\hline
FORT&4&&CRITIQUE&6\\
\hline
\end{tabular}
\end{center}
\item []
\item Env: environnement
\item RH : ressources humaines
\item P : Pascal Edouard, J : Julien Legras, C : Claire Smets, W : William Boisseleau, M : Mathieu Latimier
\end{itemize}
\section{Registre des risques}

\begin{center}
\begin{tabular}{|l|p{4cm}|p{4cm}|l|l|l|l|}
\hline
\textbf{Réf.} & \textbf{Description} & \textbf{Facteurs} & \textbf{Type} & \textbf{Probabilité} & \textbf{Impact} & \textbf{Criticité} \\
\hline
 \hline
 R01 & Absences occasionnelles (quelques heures à une journée). &
 Entretiens, Administrations, Logement, Cours à l'INSAA pour William, Désagréments. &
 Humain &
 Fort &
 Mineur &
 16 \\
 \hline
 R02 & Absences prolongées sur plusieurs jours d'un ou plusieurs membres de l'équipe. &
 Accidents, Maladies, voyages à l'étranger imprévus. &
 Humain &
 Majeur &
 Majeur &
 15 \\
 \hline
 R03 & Programme de récupération des certificats ne fonctionne pas. &
 Mauvaise implémentation, bug intempestif, aucune connexion internet. &
 Technique &
 Faible &
 Majeur &
 10 \\
\hline
 R04 & Programme de factorisation ne fonctionne pas. &
 Mauvaise implémentation, bug intempestif, ressources insuffisantes (mémoire, processeur, etc). &
 Technique &
 Fort &
 Majeur &
 20 \\
\hline
 R05 & Perte des données. &
 Panne serveur GIT, récupération impossible des données depuis le serveur. &
 Technique &
 Faible &
 Majeur &
 10 \\
\hline
 R06 & Récupération d'adresses IP impossible pour SSH. &
 Rapports d'abus. &
 Technique/Légal &
 Majeur &
 Majeur &
 15 \\
\hline

\end{tabular}
\end{center}

\section{Plans d'actions}
\begin{center}
\begin{tabular}{|l|p{7cm}|p{5cm}|p{2.5cm}|p{1.5cm}|}
\hline
\textbf{Réf.}&\textbf{Action(s) prévue(s)}&\textbf{Action effectuée}&\textbf{Date}&\textbf{Par}\\
\hline
        R01 &
        Soit l'absence ne génère pas de retard sur la tâche en cours (ou peut être rattrapée par les autres acteurs), soit la journée doit être rattrapée en cumulant plus d'heures dans la semaine.
         & Répartition des horaires de travail le jour même ou le samedi & 20/01/14 (M) 22/01/14 (W) 23/01/14 (J) 27/01/14 (M) 28/01/14 (C) & Pascal \\
\hline
        R02 &
        	Réorganisation et répartition de l'ensemble des tâches entre les membres restants de l'équipe.
         & & & Pascal\\
\hline
        R03 &
        Retravailler la structure du programme, répartir les bugs existant aux membres de l'équipe, changer de réseau. & & & Julien\\
\hline
R04 &
        Retravailler la structure du programme, répartir les bugs existant aux membres de l'équipe, discuter d'une ressource disponible pour exécuter le programme de calcule.& 
        Résolutions des bugs et comparer les résultats du programme avec les résultats attendus lors des test& 
        31/01/14 (W)
        03/02/14 (J) & Julien\\
\hline
R05 &
        Faire une évaluation des données restants sur les machines de chaque membres de l'équipe et calculer les pertes de données. & & & William et Mathieu\\
\hline
R06 & Arrêter temporairement le scan et décider avec le client de l'action à entreprendre. & Abandon de la partie SSH en accord avec le client. & 21/01/2014 & Claire\\
\hline
\end{tabular}
\end{center}
\newpage
\section{Annexes}

\subsection{Mail reçu le 19/01/2014 lié au risque R06}
\begin{verbatim}
Dear Amazon EC2 Customer,

We've received a report that your instance(s):

Instance Id: i-6394312c
IP Address: 54.194.102.0

has been port scanning remote hosts on the Internet; check the information provided below by
the abuse reporter.

This is specifically forbidden in our User Agreement: http://aws.amazon.com/agreement/

Please immediately restrict the flow of traffic from your instances(s) to cease disruption to
other networks and reply this email to send your reply of action to the original abuse
reporter. This will activate a flag in our ticketing system, letting us know that you have
acknowledged receipt of this email.

It's possible that your environment has been compromised by an external attacker. It remains
your responsibility to ensure that your instances and all applications are secured.

Case number: 11135140320-1

Additional abuse report information provided by original abuse reporter:
* Destination IPs:
* Destination Ports: 22
* Destination URLs:
* Abuse Time: Sun Jan 19 11:36:00 UTC 2014
* Log Extract:
<<<
54.194.102.0 was observed probing caltech.edu for security holes. It
has been blocked at our border routers. It may be compromised.
...
\end{verbatim}
\end{document}