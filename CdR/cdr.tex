\documentclass[a4paper,11pt,french]{article}
\usepackage[utf8]{inputenc}

\usepackage[T1]{fontenc}
\usepackage[francais]{babel} 
\usepackage[top=2cm, bottom=2cm, left=2cm, right=2cm, includeheadfoot]{geometry} %pour les marges
\usepackage{lmodern}
\usepackage{fancyhdr} % Required for custom headers
\usepackage{lastpage} % Required to determine the last page for the footer
\usepackage{extramarks} % Required for headers and footers
\usepackage{graphicx} % Required to insert images
\usepackage{tabularx, longtable}
\usepackage{color, colortbl}
\usepackage{cmbright}

\linespread{1.1} % Line spacing

% Set up the header and footer
\pagestyle{fancy}
\lhead{\textbf{\hmwkClass -- \hmwkSubject \\ \hmwkTitle \\ \hmwkDocName}} % Top left header
\rhead{\includegraphics[width=10em]{logo_univ.png}}
\lfoot{\lastxmark} % Bottom left footer
\cfoot{} % Bottom center footer
\rfoot{Page\ \thepage\ / \pageref{LastPage}} % Bottom right footer
\renewcommand\headrulewidth{0.4pt} % Size of the header rule
\renewcommand\footrulewidth{0.4pt} % Size of the footer rule

\setlength{\headheight}{40pt}

\newcommand{\hmwkTitle}{Audit des implantations SSL/TLS} % Assignment title
\newcommand{\hmwkClass}{Master 2 SSI } % Course/class
\newcommand{\hmwkAuthorName}{Boisseleau W. et Latimier M.} % Your name
\newcommand{\hmwkSubject}{Conduite de projet} % Subject
\newcommand{\hmwkDocName}{Cahier de recettes} % Document name

\newcommand{\version}{1.1} % Document version
\newcommand{\docDate}{10.12.13} % Document date
\newcommand{\checked}{-} % Checker name
\newcommand{\approved}{X} % Approver name

\definecolor{gris}{rgb}{0.95, 0.95, 0.95}

\author{\hmwkAuthorName}
\date{} % Insert date here if you want it to appear below your name



\newcommand{\fiche}[7] {
	\noindent
\begin{tabular}{|p{0.5cm}| p{6cm} | p{1cm} | p{4cm} | p{1.5cm}| p{1.5cm}|} 
\hline
\rowcolor{blue}
\multicolumn{2}{|l|}{\color{white}\bfseries{Objet testé : \color{white} \bfseries{#1} }} & 
\multicolumn{4}{l|}{\color{white}\bfseries{Version : \color{white}\bfseries{#2} }}\\
\hline
\multicolumn{6}{|l|}{\textbf{Objectif de test : #3} }\\
\hline
\multicolumn{6}{|l|}{\textbf{Procédure num. #4} }\\
\hline
\textbf{N.} & \textbf{Actions} & \multicolumn{2}{p{5cm}|}{\textbf{Résultats attendus}} & \textbf{Exig.} & \textbf{OK/NOK} \\
\hline
%#5 & #6 & \multicolumn{2}{p{6cm}|}{#7} & #8 & #9\\
#5\\
\hline
#6\\
\hline
#7\\
\hline

\end{tabular}

}



%\newcommand{\fiche2}[14] {
%	\noindent
%\begin{tabular}{|p{0.5cm}| p{6cm} | p{1cm} | p{6cm} | p{1.5cm}| p{1.5cm}|} 
%\hline
%\rowcolor{blue}
%\multicolumn{2}{|l|}{\color{white}\bfseries{Objet testé : \color{white} \bfseries{#1} }} & 
%\multicolumn{4}{l|}{\color{white}\bfseries{Version : \color{white}\bfseries{#2} }}\\
%\hline
%\multicolumn{6}{|l|}{\textbf{Objectif de test : #3} }\\
%\hline
%\multicolumn{6}{|l|}{\textbf{Procédure num. #4} }\\
%\hline
%\textbf{N.} & \textbf{Actions} & \multicolumn{2}{l|}{\textbf{Résultats attendus}} & \textbf{Exig.} & \textbf{OK/NOK} \\
%\hline
%#5 & #6 & \multicolumn{2}{l|}{#7} & #8 & #9\\
%\hline
%#10 & #11 & \multicolumn{2}{l|}{#12} & #13 & #14\\
%\hline
%\end{tabular}
%\\
%}





\begin{document}
\pagestyle{fancy}

\vspace*{5cm}
\begin{center}\textbf{\Huge{\hmwkDocName}}\end{center}
\vspace*{7cm}
	
\begin{center}
\fcolorbox{black}{gris}{
\begin{minipage}{10cm}
\begin{tabularx}{10cm}{lXl}
	\bfseries{Version} & & \version\\
	& & \\
	\bfseries{Date} & & \docDate\\
	& & \\
	\bfseries{Rédigé par} & & \hmwkAuthorName \\
	& & \\
	\bfseries{Relu par} & & \checked \\
	& & \\
	\bfseries{Approuvé par} & & \approved \\
	& & \\
\end{tabularx}
\end{minipage}
}
\end{center}

\newpage

%Tableau de mises à jour
\vspace*{1cm}
\begin{center}
\textbf{\huge{MISES À JOUR}}\\
\vspace*{3cm}
	\begin{tabularx}{16cm}{|c|c|X|}
	\hline
	\bfseries{Version} & \bfseries{Date} & \bfseries{Modifications réalisées}\\
	\hline
	1.1 & 10.12.13 & Création et complétion du cahier de recette \\
	1.2 & 10.12.13 & Complétion du cahier de recette \\
	\hline
	\end{tabularx}
\end{center}

%La table des matières
\clearpage
\tableofcontents
\clearpage

% OBJET
\section{Objet}

L'objet de ce cahier de recettes est de lister les tests praticables et pratiqués sur les différentes fonctionnalités liées au projet avant le lancement de celui-ci et sa livraison au client. 

\section{Documents applicables et de référence}

\section{Environnement de test}

\section{Responsabilités}

\section{Stratégie de tests}


% \section{Gestion des anomalies}
% Créer un dossier gestion des anomalies , en rapport ou non avec le cahier des charge.
%


\section{Procédures de test}
\subsection{ZMAP}
\subsubsection*{Terminologie}
Sont définis les acronymes suivants : 
\begin{itemize}
	\item \texttt{R} : réseau virtuel fixé représentant toutes les configurations réseau envisageables,
	\item \texttt{R'} : réseau virtuel altéré de \texttt{R},
	\item \texttt{lA} : liste d'adresses prédéfinie de machines de \texttt{R},
	\item \texttt{A1} : liste d'adresses de machines ayant des ports SSH ouverts sur \texttt{R},
	\item \texttt{A2} : liste d'adresses de machines ayant des ports SSH protégés et fermés sur \texttt{R},
	\item \texttt{A1'} : liste d'adresses de machines ayant des ports SSH ouverts sur \texttt{R'},
	\item \texttt{A2'} : liste d'adresses de machines ayant des ports SSH protégés et fermés sur \texttt{R'},
	\item \texttt{VA1} : liste d'adresses de machines attendues ayant des ports SSH ouverts sur \texttt{R},
	\item \texttt{VA2} : liste d'adresses de machines attendues ayant des ports SSH protégés et fermés sur \texttt{R},
	\item \texttt{VA1'} : liste d'adresses de machines attendues ayant des ports SSH ouverts sur \texttt{R'},
	\item \texttt{VA2'} : liste d'adresses de machines attendues ayant des ports SSH protégés et fermés sur \texttt{R'}.
\end{itemize}


\subsubsection*{Procédures}

\noindent
\begin{tabular}{|p{0.5cm}| p{6cm} | p{1cm} | p{4cm} | p{1.5cm}| p{1.5cm}|} 
\hline
\rowcolor{blue}
\multicolumn{2}{|l|}{\color{white}\bfseries{Objet testé : \color{white} \bfseries{ZMAP} }} & 
\multicolumn{4}{l|}{\color{white}\bfseries{Version : \color{white}\bfseries{1.0} }}\\
\hline
\multicolumn{6}{|l|}{\textbf{Objectif de test : ZMAP reconnaît port O/F/N-A} }\\
\hline
\multicolumn{6}{|l|}{\textbf{Procédure P1 : TestPortsZMAP(ZMAP,R,R',lA,VA1,VA2,VA1',VA2')} }\\
\hline
\textbf{N.} & \textbf{Actions} & \multicolumn{2}{p{5cm}|}{\textbf{Résultats attendus}} & \textbf{Exig.} & \textbf{OK/NOK} \\
\hline
1 & On lance ZMAP sur \texttt{R} avec \texttt{lA} en entrée  & \multicolumn{2}{p{6cm}|}{ZMAP retourne \texttt{A1} et \texttt{A2}. On vérifie que \textit{A1=VA1} et \textit{A2=VA2}. } &  & / \\
\hline
2 & On relance ZMAP sur \texttt{R'} avec \texttt{A2} en entrée  & \multicolumn{2}{p{6cm}|}{ZMAP retourne \textit{A1'} et \textit{A2'}. On vérifie que \textit{A1'=VA1'} et \textit{A2'=VA2'}. } & &  / \\
\hline
\end{tabular}



\noindent
\begin{tabular}{|p{0.5cm}| p{6cm} | p{1cm} | p{4cm} | p{1.5cm}| p{1.5cm}|} 
\hline
\rowcolor{blue}
\multicolumn{2}{|l|}{\color{white}\bfseries{Objet testé : \color{white} \bfseries{ZMAP} }} & 
\multicolumn{4}{l|}{\color{white}\bfseries{Version : \color{white}\bfseries{1.0} }}\\
\hline
\multicolumn{6}{|l|}{\textbf{Objectif de test : Tester la portabilité du résultat de ZMAP pour l'application RC} }\\
\hline
\multicolumn{6}{|l|}{\textbf{Procédure P2 : Portabilité(ZMAP,lA,R)} }\\
\hline
\textbf{N.} & \textbf{Actions} & \multicolumn{2}{p{5cm}|}{\textbf{Résultats attendus}} & \textbf{Exig.} & \textbf{OK/NOK} \\
\hline
1 & Sur le système Linux, on lance ZMAP sur \texttt{R} avec \texttt{lA} en entrée. On récupère A1 et A2. On relance ZMAP sur A1 et A2 & \multicolumn{2}{p{6cm}|}{ZMAP s'exécute correctement sur A1 à A2} & P1 & / \\
\hline
2 & Même procédé sur le système Windows & \multicolumn{2}{p{6cm}|}{ZMAP se termine correctement sur A1 à A2} & P1 & / \\
\hline
\end{tabular}


\subsection{RC}
\subsubsection*{Terminologie}
Sont définis les acronymes suivants : 
\begin{itemize}
\item \texttt{LAd} : liste d'adresses de machines ayant des ports ouverts générées par ZMAP,
\item \texttt{$M_i$} : machine $i$ de la \texttt{LAd},
\item \texttt{$C_i$} : certificat ou chaine de certification de la machine $i$ selon RC,
\item \texttt{$VC_i$} : certificat ou chaine de certification de la machine $i$ selon $M_i$,
\item \texttt{$C_i'$} : certificat ou chaine de certification de la machine $i$ selon RC,
\item \texttt{$VC_i'$} : certificat ou chaine de certification de la machine $i$ après stockage.
\end{itemize}

\subsubsection*{Procédures}
\noindent
\begin{tabular}{|p{0.5cm}| p{6cm} | p{1cm} | p{4cm} | p{1.5cm}| p{1.5cm}|} 
\hline
\rowcolor{blue}
\multicolumn{2}{|l|}{\color{white}\bfseries{Objet testé : \color{white} \bfseries{RC} }} & 
\multicolumn{4}{l|}{\color{white}\bfseries{Version : \color{white}\bfseries{1.0} }}\\
\hline
\multicolumn{6}{|l|}{\textbf{Objectif de test : Établissement de connexion avec RC} }\\
\hline
\multicolumn{6}{|l|}{\textbf{Procédure P3 : conRC(RC,lA)} }\\
\hline
\textbf{N.} & \textbf{Actions} & \multicolumn{2}{p{5cm}|}{\textbf{Résultats attendus}} & \textbf{Exig.} & \textbf{OK/NOK} \\
\hline
1 & RC lit \texttt{LAd}. & \multicolumn{2}{p{6cm}|}{RC ne retourne pas d'erreur, il reconnaît les $M_i$ comme valides} & P2 & / \\
\hline
\end{tabular}

\noindent
\begin{tabular}{|p{0.5cm}| p{6cm} | p{1cm} | p{4cm} | p{1.5cm}| p{1.5cm}|} 
\hline
\rowcolor{blue}
\multicolumn{2}{|l|}{\color{white}\bfseries{Objet testé : \color{white} \bfseries{RC} }} & 
\multicolumn{4}{l|}{\color{white}\bfseries{Version : \color{white}\bfseries{1.0} }}\\
\hline
\multicolumn{6}{|l|}{\textbf{Objectif de test : Récupération des certificats } }\\
\hline
\multicolumn{6}{|l|}{\textbf{Procédure P4 : recCertifRC(RC,lA)} }\\
\hline
\textbf{N.} & \textbf{Actions} & \multicolumn{2}{p{5cm}|}{\textbf{Résultats attendus}} & \textbf{Exig.} & \textbf{OK/NOK} \\
\hline
1 & RC échange de certificats avec $M_i$. & \multicolumn{2}{p{6cm}|}{Si $M_i$ autorise l'échange, récupération de $C_i$ et $C_i=VC_i$, sinon RC passe à $M_{i+1}$} & P3 & / \\
\hline
2 & RC stocke dans une base de données le certificat s'il existe & \multicolumn{2}{p{6cm}|}{Le certificat est stocké correctement et $C_i$ et $C_i'=VC_i'$} & P3 & / \\
\hline
\end{tabular}

\subsection{SD}
\subsubsection*{Terminologie}
Sont définis les acronymes suivants : 
\begin{itemize}
\item \texttt{B} base de données contenant des certificats uniques et des certificats identiques,
\item \texttt{BA} base de données B après suppression des doublons selon SD,
\item \texttt{VBA} base de données B sans doublons
\end{itemize}

\subsubsection*{Procédure}
\noindent
\begin{tabular}{|p{0.5cm}| p{6cm} | p{1cm} | p{4cm} | p{1.5cm}| p{1.5cm}|} 
\hline
\rowcolor{blue}
\multicolumn{2}{|l|}{\color{white}\bfseries{Objet testé : \color{white} \bfseries{SD} }} & 
\multicolumn{4}{l|}{\color{white}\bfseries{Version : \color{white}\bfseries{1.0} }}\\
\hline
\multicolumn{6}{|l|}{\textbf{Objectif de test : Tests de suppression de doublons } }\\
\hline
\multicolumn{6}{|l|}{\textbf{Procédure P4 : supD(SD,B,VBA)} }\\
\hline
\textbf{N.} & \textbf{Actions} & \multicolumn{2}{p{5cm}|}{\textbf{Résultats attendus}} & \textbf{Exig.} & \textbf{OK/NOK} \\
\hline
1 & SD s'exécute sur B & \multicolumn{2}{p{6cm}|}{SD s'exécute correctement et BA=VBA} & & / \\
\hline
\end{tabular}


\subsection{F}
\subsubsection*{Terminologie}
Sont définis les acronymes suivants : 
\begin{itemize}
\item $lA_1$ : liste de taille impaire d'entiers de même taille binaire,
\item $lA_2$ : liste de taille paire d'entiers de même taille binaire, 
\item $F1$ : section de l'algorithme de  $F$ calculant l'arbre des produits,
\item $F2$ : $F-F1$ section de l'algorithme de  $F$ calculant l'arbre des restes,
\item $lF_1$ : liste des fils de $lA_1$ calculée suivant $F1$ sur $lA_1$,
\item $lF_2$ : liste des fils de $lA_2$ calculée suivant $F1$ sur $lA_2$,
\item $VlF_1$ : liste des fils attendue pour $lA_1$,
\item $VlF_2$ : liste des fils attendue pour $lA_2$,
\item $P$ : entier produit de facteurs après lancement de $F1$ sur $lA_A$,
\item $modP1$ : élément gauche après calcul du modulo de $P$ suivant $F2$,
\item $modP2$ : élément droite après calcul du modulo de $P$ suivant $F2$,
\item $VmodP1$ : élément gauche modulo de $P$ suivant $F2$,
\item $VmodP2$ : élément droite modulo de $P$ suivant $F2$.
\end{itemize}

\subsubsection*{Procédures}
\noindent
\begin{tabular}{|p{0.5cm}| p{6cm} | p{1cm} | p{4cm} | p{1.5cm}| p{1.5cm}|} 
\hline
\rowcolor{blue}
\multicolumn{2}{|l|}{\color{white}\bfseries{Objet testé : \color{white} \bfseries{F} }} & 
\multicolumn{4}{l|}{\color{white}\bfseries{Version : \color{white}\bfseries{1.0} }}\\
\hline
\multicolumn{6}{|l|}{\textbf{Objectif de test : Calcul des fils selon $F1$ suivant une liste} }\\
\hline
\multicolumn{6}{|l|}{\textbf{Procédure P5 : calcF1($F1$,$lA_1$,$lA_2$,$VlF_1$,$VlF_2$)} }\\
\hline
\textbf{N.} & \textbf{Actions} & \multicolumn{2}{p{5cm}|}{\textbf{Résultats attendus}} & \textbf{Exig.} & \textbf{OK/NOK} \\
\hline
1 & On calcule $F1(lA_1)=lF_1$ & \multicolumn{2}{p{6cm}|}{$F1$ s'exécute correctement et $lF_1=VlF_1$} & & / \\
\hline
2 & On calcule $F1(lA_2)=lF_2$ & \multicolumn{2}{p{6cm}|}{$F1$ s'exécute correctement et $lF_2=VlF_2$} & & / \\
\hline
\end{tabular}

\noindent
\begin{tabular}{|p{0.5cm}| p{6cm} | p{1cm} | p{4cm} | p{1.5cm}| p{1.5cm}|} 
\hline
\rowcolor{blue}
\multicolumn{2}{|l|}{\color{white}\bfseries{Objet testé : \color{white} \bfseries{F} }} & 
\multicolumn{4}{l|}{\color{white}\bfseries{Version : \color{white}\bfseries{1.0} }}\\
\hline
\multicolumn{6}{|l|}{\textbf{Objectif de test : Calcul des modulos fils avec  $F2$ suivant P} }\\
\hline
\multicolumn{6}{|l|}{\textbf{Procédure P6 : calcF2($F2$,$P$,$VmodP1$,$VmodP2$)} }\\
\hline
\textbf{N.} & \textbf{Actions} & \multicolumn{2}{p{5cm}|}{\textbf{Résultats attendus}} & \textbf{Exig.} & \textbf{OK/NOK} \\
\hline
1 & On calcule $F2(P)=(modP1,modP2)$ & \multicolumn{2}{p{6cm}|}{$F2$ s'exécute correctement et $modP1=VmodP1$ et $modP2=VmodP2$} & P5 & / \\
\hline

\end{tabular}

\end{document}
